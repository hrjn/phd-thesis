\chapter{Calcul des matrices source-récepteur par un modèle lagrangien adjoint}

L'étude de cas sur l'expérience FFT07 présentée dans le Chapitre 3 a montré que la charge de calcul de la procédure d'estimation du terme source est en grande majorité focalisée sur le calcul de la matrice source-récepteur. Cette étape nécessite une quantité d'appels au modèle de dispersion égale au nombre de particules échantillonnées par itération multiplié par le nombre d'itérations, soit en pratique 1000 instances dans le cas de FFT07. \\

Si pour les cas les plus simples il est possible de se limiter à des modèles de dispersion suffisamment rapides, la situation change dès qu'il s'agit de considérer un contexte plus élaboré, avec des outils de calculs permettant certes une meilleure représentation des phénomènes de dispersion, mais dont l'exécution devient bien plus coûteuse en termes d'implémentation et de temps de calcul. \\

L'objet de ce chapitre est ainsi de présenter une façon d'optimiser le calcul des matrices source-récepteur lorsque celui-ci doit faire appel à un code de dispersion plus complexe que le modèle à bouffées gaussiennes précédemment évoqué. La présentation de cet outil de calcul constituera la première partie de ce chapitre, avec notamment une explication de son fonctionnement, puis une présentation de son intégration optimale à la chaîne de calcul et d'estimation existante. La suite du chapitre se concentrera sur les aspects pratiques autour de la mise en application de cette méthodologie améliorée, avec la présentation des résultats sur plusieurs cas-tests de simulation.


\section{Le code de calcul PMSS}

\subsection{L'approche lagrangienne de la dispersion atmosphérique}
\label{part_lagrangian}

Pour les modèles dits eulériens, la résolution du problème de la dispersion d'un polluant dans l'atmosphère passe par la construction d'un maillage sur le domaine étudié, afin de pouvoir observer l'évolution des concentrations du polluant porté par les mouvements de l'air. 

Le point de vue lagrangien est différent: il s'agit ici de résoudre un système d'équations dans un repère lié au déplacement de la masse d'air contenant le polluant. Pour cela, on représente le panache sous la forme d'un ensemble de \textit{particules lagrangiennes}\footnote{Afin d'éviter toute confusion avec les particules statistiques dont il est fait référence dans le cadre de l'algorithme AMIS, nous utiliserons la dénomination  de \textit{particule lagrangienne} abréviée par PL dans la suite du texte. Nous conservons le terme de \textit{particule} pour désigner les échantillons issus de l'AMIS.}, chacune étant porteuse d'une masse élémentaire du polluant considéré. Le principe d'un modèle lagrangien consiste ainsi à étudier les trajectoires de ces éléments discrets dans le domaine au fil du temps.

Le fait de modéliser le panache par un ensemble de PL permet de tenir compte de la nature stochastique de leur déplacement, qui traduit la variabilité inhérente aux processus de turbulences auxquels est soumis le panache: on parle d'ailleurs plus précisément de \textit{modèle lagrangien stochastique}. On va ainsi travailler sur une équation de transport portant sur la densité de probabilité associée à chaque trajectoire. Plus formellement, d'après \cite{Flesch1995}, la formulation classique régissant un modèle de dispersion lagrangien se présente sous la forme d'une équation de Langevin, qui s'écrit: 

\begin{equation}
	\begin{split}
		du_i &= a_i(\Vecx, \Vecu, t)dt + b_{i,j}(\Vecx, \Vecu, t)d\xi_j  \\
		dx_i &= u_i dt = (\bar{u}_i + U_i)dt
	\end{split}
	\label{eq_langevin}
\end{equation}
où:
 \begin{itemize}
	\item $\Vecx = (x, y, z)$ est la position de la PL  définie par un repère spécifique: $x$ suit l'axe du vent, $y$ suit l'axe perpendiculaire au vent, et $z$ désigne l'élévation verticale classique. 
	\item $\Vecu$ est la vitesse d'écoulement à laquelle est soumise la PL: $\Vecu = (u, v, w)$ où les composantes respectives de ce vecteur suivent les mêmes axes que $\Vecx$.
	\item $a_i$ et $b_{i,j}$ sont des fonctions spécifiques de $(\Vecx, \Vecu, t)$ respectivement appelées \textit{drift term} et \textit{random forcing}.
	\item $d\xi_j$ est un incrément aléatoire suivant une distribution gaussienne de moyenne nulle et de variance $dt$.
	\item $\bar{u}_i$ représente le vent moyen et $U_i$ sa composante stochastique.
	
\end{itemize}

L'expression des fonctions $a_i$ et $b_{i,j}$ varie selon les hypothèses que l'on se fixe sur la nature de la turbulence: une présentation plus détaillée de leur calcul est disponible dans \cite{Wilson1996}. Une fois que ceux-ci sont définis,l'équation \eqref{eq_langevin} est discrétisée et sa résolution permet de calculer un ensemble de trajectoires de PL émanant d'une source dont les paramètres sont connus. Les concentrations volumiques moyennes simulées sont alors obtenues par la somme des particules présentes dans un volume élémentaire $d\Vecx$ autour du point d'observation $\Vecx$ durant un certain temps de résidence. En d'autres termes, on peut écrire la concentration moyenne au point $\Vecx$ et à l'instant $t$ comme étant : 

\begin{equation}
	C(\Vecx, t) = \int _{-\infty}^{t} \int_{V} S(\Vecx',t')p(\Vecx, t | \Vecx', t')d\Vecx'dt
	\label{eq_c_moyen_lagrangien}
\end{equation}
où $V$ est le volume défini par le domaine d'étude, $S(\Vecx',t')$ est la distribution de la source, et $p(\Vecx, t | \Vecx', t')$ est la densité de probabilité sur la position $\Vecx$ et l'instant $t$ des PL de position initiale $\Vecx'$ à l'instant $t'$. \\

\begin{figure}
	\centering
	\includegraphics[width=0.65\textwidth]{lagrangian_ok}
	\caption{Principe du modèle lagrangien: la concentration en $\Vecx$ s'obtient par la somme des PL (en vert) traversant le volume élémentaire $d\Vecx$ durant un certain temps de résidence.}
	\label{fig_schema_lagrangien}
\end{figure}

Afin de calculer le champ d'écoulement auquel sont soumises les PL, plusieurs méthodes sont disponibles. Il est par exemple possible de résoudre les équations de la mécanique des fluides via un outil de simulation de type CFD, ce qui permet une modélisation fine des phénomènes physiques mis en jeu. Une autre possibilité consiste à avoir recours à une simulation dite \textit{CFD simplifiée}, où le champ de vent est interpolé à partir des mesures d'une ou plusieurs stations météorologiques tout en prenant en compte la topographie du terrain. C'est cette dernière approche qui est employée dans les outils de calcul du CEA et d'ARIA Technologies, et que nous présentons dans le chapitre suivant. \\

\subsection{La chaîne de calcul SWIFT-SPRAY}

L'outil \textit{Parallel Micro-SWIFT-SPRAY} (PMSS) est une chaîne de calcul constituée de deux éléments distincts: un outil de CFD simplifiée (SWIFT) et un modèle de dispersion lagrangien stochastique (SPRAY). Il est généralement appliqué dans des études à petite échelle (par exemple, au niveau d'un quartier), mais grâce à sa version parallèle, il a récemment été utilisé sur des domaines plus grands, à l'exemple du cas AirCity (REF) où le modèle a été exécuté sur l'ensemble de la ville de Paris.\\

\subsubsection{SWIFT}

Le modèle SWIFT permet de produire des champs de vent 3D en exploitant différents types de données météorologiques sur un même site (profils de vent et de température, stations de mesures, sorties de modèles météorologiques de prévision). Il permet notamment de prendre en compte la topographie du milieu, la présence d'obstacles tels que des bâtiments, l'occupation des sols ou encore l'influence de la stabilité atmosphérique. Son fonctionnement peut être résumé en quatre étapes :  \\

\begin{enumerate}
	\item Dans un premier temps, les mesures météorologiques reçues en entrée sont interpolées sur les différents points constituant une version discrétisée du domaine.
	\item Dans un deuxième temps, l'effet des obstacles présents dans le domaine sur l'écoulement sont modélisés via la création de zones spécifiques dans le voisinage de ces obstacles où le champ de vitesse est calculé de façon spécifique.
	\item La troisième étape consiste à ajuster le champ de vent en appliquant un principe de conservation de la masse.
	\item Enfin, la dernière étape consiste à calculer la turbulence intrinsèque à l'écoulement modélisé.\\
\end{enumerate}

\begin{figure}[h!]
	\centering
	\includegraphics[width=0.75\textwidth]{swift_exemple}
	\caption{Exemple de calcul d'un champ de vent autour d'un obstacle avec SWIFT, avant (à gauche) et après (à droite) l'ajustement du champ}
	\label{fig_swift_exemple}
\end{figure}
En sortie de cet enchaînement de calculs, on obtient un champ de vent 3D qui peut alors directement être exploité par le modèle de dispersion SPRAY.



\subsubsection{SPRAY}

SPRAY est un modèle de dispersion lagrangien stochastique dont les principes de base suivent les mécanismes présentés à la section \ref{part_lagrangian}. L'implémentation de SPRAY repose sur le critère dit de \textit{well-mixed condition} permettant de donner une formulation explicite aux termes $a_i$ et $b_{i,j}$ de l'équation \eqref{eq_langevin} , et présenté en détail dans les travaux de \cite{Thomson1987}.

En pratique, plusieurs fonctionnalités supplémentaires sont implémentées dans SPRAY, telles que: \\

\begin{itemize}
	\item le "rebond" des particules sur les obstacles,
	\item le calcul de doses pour les sources radioactives,
	\item la prise en compte des différents types de dépôts (secs ou humides).\\
\end{itemize}

\begin{figure}[h!]
	\centering
	\includegraphics[width=0.65\textwidth]{spray_exemple}
	\caption{Exemple de champ de concentration calculé par SPRAY dans un domaine de type urbain}
	\label{fig_spray_exemple}
\end{figure}

La combinaison de SWIFT et SPRAY permet ainsi de calculer un champ de concentration sur le domaine étudié, connaissant les paramètres du terme source qui sont soumis en entrée du modèle SPRAY. Dans le contexte de ce chapitre, la chaîne de calcul PMSS permet de:

\begin{itemize}
	\item générer un jeu d'observations synthétiques en simulant un rejet induit par une source que l'on va chercher à retrouver: pour cela, on calcule les valeurs du champ de concentrations en un nombre fini de points du domaine que nous définirons comme étant les observations fournies par les capteurs,
	\item construire les matrices source-récepteur lors de l'exécution de l'algorithme AMIS, nécessaires au processus d'estimation du terme source.\\
\end{itemize}

\subsection{Dualité \textit{forward-backward}}

L'optimisation du calcul des matrices source-récepteur suivant le modèle de l'équation \eqref{eq_AE_4} est un point important: en effet dans une approche directe telle que présentée dans le chapitre précédent, la construction des $\MatC(\VecTheta)$ pour chaque particule $\VecTheta$ échantillonnée depuis la loi de proposition courante fait appel à autant de calculs de dispersion, ce qui rend l'opération d'estimation du terme source très coûteuse en temps de calcul.\\

Nous proposons dans la suite de ce chapitre une amélioration du processus d'estimation privilégiant le calcul des matrices source-récepteur par une approche de type \textit{backward}. Cette méthodologie a initialement été introduite dans \cite{Keats2007} pour ensuite être appliquée sur un algorithme d'estimation de type MCMC, nous en rappelons les bases ci-après.\\

Soit une source ponctuelle $Q$ située au point de l'espace $\PosSource$, de débit massique constant $q_s$ et de temps d'activation et d'arrêt respectifs $t_{on}$ et $t_{off}$ définie par la distribution suivante:\\

\begin{equation}
	Q = q_s \delta(\Vecx - \PosSource)\left[H(t - t_{on}) - H(t - t_{off})\right]
\end{equation}
où $\delta$ est la distribution de Dirac et $H$ la fonction de Heaviside. On note $C$ le champ de concentration moyen induit par cette  source. La valeur simulée de concentration $c_i$ obtenue sur le $i$-ème capteur à l'instant $t$ peut se modéliser par l'équation suivante:

\begin{equation}
c_i = \int_0^t \int_V C h ~dtdV
\label{eq_int_direct}
\end{equation}
où $h$ est la \textit{fonction} de réponse du capteur. Cette équation peut se simplifier sous la notation suivante:\\

\begin{equation}
c_i = \langle	C,h\rangle
\label{eq_scal_direct}
\end{equation} 

La relation de \textit{dualité forward-backward} présentée par \cite{Keats2007} stipule que cette même concentration peut s'écrire sous la forme:

\begin{equation}
c_i = \int_0^t \int_V Q C^* ~ dtdV = \langle Q, C^*\rangle
\label{eq_int_adjoint}
\end{equation}
où $C^*$ est le champ de rétro-concentrations induit par le $i$-ème capteur, et dont les valeurs sont obtenues par la résolution de l'équation \eqref{eqn_advection_diffusion_backward}. Dans la pratique, le champ $C^*$ est simulé par un \textit{modèle de dispersion dual}, ou \textit{modèle de rétro-dispersion}.\\

La relation de dualité $\langle C,h\rangle = \langle Q,C^* \rangle$ est ainsi valable si l'équation adjointe d'advection-diffusion a été résolue de telle sorte que les conditions aux limites permettent d'annuler les termes de bord pouvant y apparaître. De plus:

\begin{enumerate}
	\item sous l'hypothèse d'un capteur idéal, sa résolution est infinie, ce qui se traduit en pratique par $h$ prenant la forme d'une distribution de Dirac,
	\item dans le cadre de la construction des matrices source-récepteur, $Q$ illustre une source instantanée et dont le débit de rejet est unitaire.
\end{enumerate}

Dans ce cas particulier, on obtient alors une équivalence entre $C$ et $C^*$: 

\begin{equation}
C \simeq C^*
\label{eq_equivalence}
\end{equation}


\subsection{Intégration d'un modèle \textit{backward} au processus d'estimation}

\subsubsection{Utilisation de RetroSPRAY dans l'AMIS}

L'outil PMSS dispose d'une version \textit{backward} de SPRAY appelée RetroSPRAY, dont l'intégration dans la chaîne de calcul se fait de la même façon que pour SPRAY. Le modèle RetroSPRAY procède à la résolution des équations de Langevin en mode inverse: 

\begin{equation}
\begin{split}
du_i^b &= U_i^b (t-dt) - U_i^b(t) \\
dx_i^b &= x_i^b (t-dt) - x_i^b(t)
\end{split}
\label{eq_langevin_inv1}
\end{equation}

On peut alors écrire l'équivalent inverse de l'équation \eqref{eq_langevin} sous la forme suivante:

\begin{equation}
	\begin{split}
	du_i^b &= a_i^b dt + b_{i,j}^b d\xi_j \\
	dx_i^b &= -(\bar{u}_i + U_i^b )dt
	\end{split}
	\label{eq_langevin_inv2}
\end{equation}

Les termes $a_i^b$ et $ b_{i,j}^b$ peuvent être calculés selon les expressions fournies dans \cite{Flesch1995} et \cite{Wilson2009}. \\

Si on examine plus en détail la construction de la matrice source-récepteur, on peut réécrire l'équation \eqref{eq_AE_4} sous une forme plus explicite: notons $C(R_i,t_j |\VecTheta, t'_n)$ la concentration moyenne au capteur $R_i$ à l'instant $t_j$ résultant d'une source située à la position $\VecTheta$ et ayant émis un rejet unitaire à l'instant $t'_n$. La version \textit{forward} de la matrice source-récepteur pour $\VecTheta$ s'écrit:

\begin{equation}
\MatC^f(\VecTheta) = 
\begin{pmatrix}
C(R_1,t_1 | \VecTheta, t'_1) & C(R_1,t_1 | \VecTheta, t'_2) & \cdots & C(R_1, t_1 |\VecTheta, t'_{T_s}) \\
C(R_1,t_2 | \VecTheta, t'_1) & C(R_1,t_2 | \VecTheta, t'_2) & \cdots & C(R_1, t_2 |\VecTheta, t'_{T_s}) \\
\vdots & \vdots & & \vdots \\
C(R_1,t_{T_c} | \VecTheta, t'_1) & C(R_1,t_{T_c} | \VecTheta, t'_2) & \cdots & C(R_1, t_{T_c} |\VecTheta, t'_{T_s}) \\
C(R_2,t_1| \VecTheta, t'_1) & C(R_2,t_1 | \VecTheta, t'_2) & \cdots & C(R_2, t_1 |\VecTheta, t'_{T_s}) \\
\vdots & \vdots & & \vdots \\
\vdots & \vdots & & \vdots \\
C(R_{N_c},t_{T_c} | \VecTheta, t'_1) & C(R_{N_c},t_{T_c} | \VecTheta, t'_2) & \cdots & C(R_{N_c}, t_{T_c} |\VecTheta, t'_{T_s}) \\
\end{pmatrix}
\label{eq_matrix_forward}
\end{equation}

En appliquant la relation de dualité \textit{forward-backward}, dans l'espace dual où le modèle \textit{backward} opère, les sources deviennent des "rétro-capteurs", et les capteurs deviennent des "rétro-sources". On définit alors $C^*(\VecTheta,t'_n | R_i, t_j)$ comme la rétro-concentration mesurée au point $\VecTheta$ à l'instant $t'_n$ provenant d'une rétro-source située à la position $R_i$ et ayant émis un rétro-rejet unitaire à l'instant $t_j$. La version \textit{backward} de $\MatC^f$ s'écrit alors:

\begin{equation}
\MatC^b (\VecTheta)= 
\begin{pmatrix}
	C^*(\VecTheta, t'_1 | R_1, t_1) & C^*(\VecTheta, t'_2 | R_1, t_1) & \cdots & C^*(\VecTheta, t'_{T_s} | R_1, t_1) \\
	C^*(\VecTheta, t'_1 | R_1, t_2) & C^*(\VecTheta, t'_2 | R_1, t_2) & \cdots & C^*(\VecTheta, t'_{T_s} | R_1, t_1) \\
	\vdots & \vdots & & \vdots \\
	C^*(\VecTheta, t'_1| R_1,t_{T_c}) & C^*(\VecTheta, t'_2 | R_1, t_{T_c}) & \cdots & C^*(\VecTheta, t'_{T_s} | R_1, t_{T_c}) \\ 
	\vdots & \vdots & & \vdots \\
	\vdots & \vdots & & \vdots \\
	C^*(\VecTheta, t'_1| R_{N_c},t_{T_c}) & C^*(\VecTheta, t'_2 | R_{N_c}, t_{T_c}) & \cdots & C^*(\VecTheta, t'_{T_s} | R_{N_c}, t_{T_c}) \\ 
	
\end{pmatrix}
\label{eq_matrix_backward}
\end{equation}

Dans l'algorithme AMIS, au moment de calculer la vraisemblance de chaque particule échantillonnée, on peut ainsi faire désormais intervenir $\MatC^b$ comme matrice source-récepteur.

\subsubsection{Avantages}

Le fait de substituer un modèle \textit{backward} au modèle direct permet de n'avoir à faire qu'un seul calcul de rétro-dispersion par capteur, qui donne alors l'ensemble des valeurs de rétro-concentrations sur le domaine. De plus, ces calculs sont désormais opérés en amont du schéma itératif de l'AMIS: au lieu d'exécuter une boucle qui lance des calculs de dispersion pour chaque particule AMIS et à chaque itération, les champs $C^*$ sont pré-calculés et déjà disponibles au moment de l'estimation du terme source. \\

\begin{figure}[h!]
	\centering
	\includegraphics[width=0.75\textwidth]{schema_amis_optimise}
	\caption{Schéma de principe de la version \textit{backward} de l'algorithme d'estimation du terme source}
	\label{fig_schema_amis_optimise}
\end{figure}

Comme l'illustre la figure \ref{fig_schema_amis_optimise}, aucun calcul de dispersion n'est donc instancié durant la mise en oeuvre de l'algorithme, car les matrices source-récepteur sont désormais construites grâce à des opérations de lecture des fichiers dans lesquels ont été stockés les valeurs de $C^*$ pré-calculées. PMSS permet en effet d'agréger les résultats des calculs de dispersion dans des fichiers binaires, qui servent de base de données où les valeurs de rétro-concentration en chaque point du domaine peuvent être lues par un module d'entrée/sortie intégré à l'implémentation de l'AMIS.

\subsubsection{Limitations}

Il est toutefois important de rappeler que le fait d'associer le modèle \textit{backward} au modèle direct constitue une approximation. En effet, la partie "diffusion" du processus de dispersion atmosphérique est aléatoire, et ne peut être parfaitement reproduite dans le cas dual pour une configuration identique du modèle direct, créant ainsi un écart incompressible entre les champs $C$ et $C^*$. 

Afin de réduire au mieux cet écart, il convient d'utiliser un nombre de PL suffisamment grand pour modéliser les rétro-rejets, en prenant toutefois garde à ne pas choisir une grandeur trop élevée, qui demanderait un temps de calcul trop important au modèle. En prenant cela en compte, il demeure nécessaire de quantifier cet écart. Pour cela, un test préliminaire a été effectué, afin de comparer les résultats obtenus par les deux approches (directe et \textit{backward}) lorsqu'il s'agit de simuler des données d'observation synthétiques. Concrètement, trois constructions du vecteur $\VecObs$ ont été étudiées:

\begin{enumerate}
	\item en spécifiant directement à SPRAY les paramètres temporels du rejet (débit),
	\item en calculant une matrice source-récepteur avec SPRAY (modèle direct), puis en multipliant cette matrice par un vecteur de débit $\VecQSource$, de façon similaire au modèle de données de l'équation \eqref{eq_AE_3} mais sans ajout de bruit,
	\item en reproduisant l'opération présentée en 2. mais avec une matrice source-récepteur obtenue par RetroSPRAY (modèle \textit{backward}).
\end{enumerate} 

\begin{figure}[h!]
	\centering
	\includegraphics[width=0.8\textwidth]{comparaison_3_obs.png}
	\caption{Comparaison des observations synthétiques générées par une approche directe (en bleu et rouge) et en \textit{backward} (en vert)}
	\label{fig_comparaison_3_obs}
\end{figure}

La figure \ref{fig_comparaison_3_obs} montre que les observations générées par l'approche \textit{backward} présentent un écart assez important avec les grandeurs équivalentes obtenues par la méthode directe. Afin de ne pas propager cette erreur dans l'estimation du terme source, nous avons choisi de n'utiliser que les observations obtenues par simulation \textit{backward} dans les cas pratiques illustrés par les prochains paragraphes.\\




\section{Exemple d'application en rase campagne}

Dans cette section nous présentons un première application sur une situation  simple dans un contexte non-urbain.

\subsection{Présentation du cas-test}

Nous considérons ici un cas-test en milieu rural reproduisant une émission accidentelle depuis un site industriel dans une zone située près de la commune de Beaune, en Bourgogne, en présence d'une topographie réelle. 

\begin{figure}[h!]
	\centering
	\includegraphics[width=0.8\textwidth]{beaune_relief_capteurs}
	\caption{Superposition du relief, de l'emplacement des capteurs et de la source du cas-test Beaune}
	\label{fig_beaune_relief}
\end{figure}

\subsubsection{Caractéristiques du domaine}
Le domaine considéré couvre une surface de \SI{6}{\square\kilo\meter} avec une source unique et un réseau relativement dense de 25 capteurs disposés en quinconce et couvrant toute la superficie du domaine (figure \ref{fig_beaune_relief}).

Pour la simulation, le domaine est discrétisé en une grille de $300 \times 300$ mailles, avec une résolution du maillage en $x$ et $y$ de \SI{20}{\meter}.

\subsubsection{Paramètres météorologiques}
Sur toute la durée de la simulation, nous considérons un vent constant en vitesse (\SI{1.5}{\m\per\second}) et en direction ($330\degres$). Le champ de vent produit par SWIFT est ainsi relativement homogène, comme observé en figure \ref{fig_beaune_vent}).

\begin{figure}[h!]
	\centering
	\includegraphics[width=0.8\textwidth]{beaune_vent}
	\caption{Champ de vent calculé par SWIFT pour le cas-test Beaune}
	\label{fig_beaune_vent}
\end{figure}


\subsubsection{Paramètres relatifs à la source}
On choisit ici de modéliser une source située dans la partie nord (voir figure \ref{fig_beaune_relief}) afin que le panache résultant couvre une partie suffisante du domaine. La source est située à une altitude de \SI{10}{\m}, et émet un rejet unique entre 10h15 et 11h, avec un débit constant de 1850 unités/s. Pour la simulation, cette source est modélisée comme étant un volume de \SI{20}{\m} $\times$ \SI{20}{\m} $\times$ \SI{10}{\m}. 

\subsubsection{Paramètres relatifs aux capteurs}
On considère un réseau de 25 capteurs disposés de façon à couvrir tout le domaine, et placés à une hauteur de \SI{10}{\meter}, égale à celle de la source. 

\begin{figure}[h!]
	\centering
	\includegraphics[width=0.8\textwidth]{observations_25CAPTEURS.png}
	\caption{Concentrations mesurées aux capteurs}
	\label{fig_observations_25CAPTEURS}
\end{figure}

Chaque capteur est modélisé par un volume de \SI{15}{\meter} $\times$ \SI{15}{\meter} $\times$ \SI{10}{\meter}, et fournit des observations de concentrations moyennes entre 10h05 et 12h00, avec une plage de moyennage de 5 minutes. La figure \ref{fig_observations_25CAPTEURS} représente les séries temporelles des concentrations moyennes relevées aux capteurs impactés par la source. 

\subsubsection{Tableau récapitulatif}

\begin{center}
	\begin{tabular}{| l || c | }
		\hline			
		Coordonnées point SO (km) & $(642000;647980)$  \\
		\hline
		Coordonnées point NE (km) & $(5204000;5209980)$ \\
		\hline
		Nombre de mailles $(x,y)$ & $(300;300)$ \\
		\hline
		Résolution maille (m) & $20$ \\
		\hline
		Vitesse vent (\SI{}{\meter\per\second}) & 1.5 \\
		\hline
		Direction vent ($\degres$) & 330 \\
		\hline
		Durée du rejet (\SI{}{\minute}) & 45 \\
		\hline
		Débit du rejet (unités/s) & 1850 \\
		\hline
		Fenêtre d'observation & De 10h05 à 12h00 \\
		\hline
	\end{tabular}
\end{center}

\subsubsection{Premiers résultats}

Avec une initialisation uniforme sur le domaine et un total de 10 itérations, l'algorithme d'estimation du terme source parvient à fournir une bonne estimation des paramètres, aussi bien concernant la localisation de la source (FIG) que la reconstruction de son profil d'émission (FIG).\\




\subsection{Influence de la variance d'observation $\varObs$}

\subsection{Analyse paramétrique sur $\varQ$}

\subsection{Modularité du réseau de capteurs}

\section{Exemple d'application à un cas urbain}

On considère à présent le cas d'un rejet bref dans un cadre urbain.\\

\subsection{Présentation du cas-test}

On se place à l'échelle d'un quartier, en l'occurrence celui de la Place de l'Opéra, à Paris: il s'agit ici d'une situation plus complexe par rapport au cas précédent, à cause de la présence d'obstacles multiples et de géométries variées sur le domaine (bâtiments). \\

\missingfigure{Opéra: domaine et capteurs}

\subsubsection{Caractéristiques du domaine}

On se place sur un domaine de 808m $\times$ 882m, soit environ $844\text{m}^2$. La discrétisation choisie est plus fine que dans le cas précédent, car la topographie est plus complexe: le domaine est ainsi divisé en 404 mailles selon $x$ et 441 mailles selon $y$, avec une résolution de 2m sur chaque direction. \\

\missingfigure{Schéma du domaine avec mise en évidence de l'aspect 3D}

\subsubsection{Paramètres météorologiques}

On choisit un vent de vitesse constante (3m/s) mais dont la direction change toutes les heures : \\

\begin{center}
\begin{tabular}{cccc}
	\centering
	Heure & 11:00:00 &  12:00:00 &  13:00:00\\ 
	\hline
	 Direction vent & $230\degres$ & $180\degres$ & $45\degres$
\end{tabular} 
\end{center}

La combinaison de ces variations temporelles ainsi que de la présence d'obstacles sur le domaine fait que les champs de vent 3D diagnostiqués par SWIFT sont relativement complexes, comme l'atteste la figure FIG.

\missingfigure{Champs de vent à 2m }

\subsubsection{Paramètres relatifs à la source}

On se place dans le cas d'une source unique située à 2m du sol, et qui émet un rejet bref d'une durée de 10 minutes entre 12h10 et 12h20, avec un débit constant de $10^4$ unités/s. Dans la simulation, cette source est modélisée par un volume de 4M $\times$ 4m $\times$ 2m. Ces caractéristiques se rapprochent de celles d'un rejet d'origine malveillante, par exemple suite à l'explosion d'une "bombe sale". \\

\missingfigure{Illustration du rejet 12h15 --> 12h40}

\subsubsection{Paramètres relatifs aux capteurs}

Le domaine contient un réseau de 10 capteurs, qui sont placés aux centres de diverses intersections de rues, ainsi que sur des places publiques. Ils sont situés à la même hauteur que la source, et leur volume de contrôle est de 4m $\times$ 4m $\times$ 2m. Ils délivrent des valeurs de concentrations moyennées sur des plages de 5 minutes. \\

\subsubsection{Tableau récapitulatif}




\subsection{Initialisation optimisée de la loi de proposition}

Dans une situation complexe comme celle du cas Opéra, il peut se révéler utile d'initialiser la loi de proposition pour l'AMIS de façon plus judicieuse qu'une simple hypothèse de répartition uniforme des particules sur le domaine. Une initialisation optimisée permettra ainsi d'explorer des zones potentiellement intéressantes plus rapidement, augmentant l'efficacité de l'algorithme d'estimation. \\

Comme dans notre cas on a choisi une loi de proposition de type mixture de $D$ gaussiennes $\varphi_1, \dots, \varphi_D$, on va chercher à estimer les moyennes $\left(\VecMu_d\right)_{1:d}$ et les matrices de covariance $\left(\MatSigma_d\right)_{1:d}$ de chacune de ces composantes, ainsi que leurs facteurs d'influence $\left(\alpha_d\right)_{1:d}$. 

Pour cela, in utilise les résultats issus d'un \textit{run} de rétro-propagation: suivant un modèle de dispersion adjoint, on construit une série de cartes des concentrations conjuguées sur tout le domaine, en transformant les capteurs en rétro-sources et en utilisant les concentrations mesurées comme valeurs de rétro-émission. Une fois ces cartes créées, elles sont vues comme des ébauches des densités de probabilité sur la position de la source pour différents temps d'émission. L'objectif est alors de caler les paramètres $\left(\alpha_d, \VecMu_d,\MatSigma_d \right)_{1:D}$ sur ces densités.\\

On part de l'instant d'observation $t_0$ qui est celui où la concentration la plus élevée a été observée. On définit ensuite un instant $t_{RP}$ qui correspond à l'instant final de la rétro-propagation, avec $t_{RP} < t_0$. 

\subsection{Résultats}





