\chapter{{AMIS et modèle lagrangien rétrograde}}

La méthodologie utilisée dans le Chapitre 3 introduit une dépendance directe (linéaire) du nombre d'appels au modèle de dispersion avec la quantité de particules échantillonnées. Pour les cas les plus simples, il est possible de se limiter à des modèles de dispersion suffisamment rapides, mais la situation change dès qu'il s'agit de considérer un contexte plus élaboré (par exemple en présence d'obstacles). Il devient en effet nécessaire d'utiliser des outils de simulation permettant une meilleure représentation des phénomènes de dispersion. Cependant, le nombre et la complexité des calculs supplémentaires effectués au sein de ces outils rallongent le temps d'exécution du modèle de dispersion, et par conséquent réduisent la vitesse à laquelle l'opération de reconstruction de la source est accomplie.\\

Dans ce chapitre, nous proposons une modification de l'approche initiale, en remplaçant l'utilisation d'un modèle de dispersion direct par celle d'un modèle rétrograde. Cette substitution permet à la fois:\\
	\begin{itemize}
		\item d'optimiser le calcul des matrices source-récepteur, afin de permettre l'utilisation d'outils de simulation appropriés à la complexité de la situation, sans toutefois augmenter de façon excessive les temps de calcul;
		\item d'exploiter des résultats de rétro-propagation pour construire une procédure d'initialisation améliorée de la loi de proposition de l'AMIS. \\
	\end{itemize}


\section{Le système de modélisation PMSS}

\subsection{L'approche lagrangienne de la dispersion atmosphérique}
\label{part_lagrangian}

Pour les modèles dits eulériens, la résolution du problème de la dispersion d'un polluant dans l'atmosphère passe par la construction d'un maillage sur le domaine étudié, afin de pouvoir observer l'évolution des concentrations du polluant porté par les mouvements de l'air. 

Le point de vue lagrangien est différent: il s'agit ici de résoudre un système d'équations dans un repère lié au déplacement de la masse d'air contenant le polluant. Pour cela, on représente le panache sous la forme d'un ensemble de \textit{particules lagrangiennes}\footnote{Afin d'éviter toute confusion avec les particules statistiques dont il est fait référence dans le cadre de l'algorithme AMIS, nous utiliserons la dénomination  de \textit{particule lagrangienne} abréviée par PL dans la suite du texte. Nous conservons le terme de \textit{particule} pour désigner les échantillons issus de l'AMIS.}, chacune étant porteuse d'une masse élémentaire du polluant considéré. Le principe d'un modèle lagrangien consiste ainsi à étudier les trajectoires de ces éléments discrets dans le domaine au fil du temps.

Le fait de modéliser le panache par un ensemble de PL permet de tenir compte de la nature stochastique de leur déplacement, qui traduit la variabilité inhérente aux processus de turbulences auxquels est soumis le panache: on parle d'ailleurs plus précisément de \textit{modèle lagrangien stochastique}. On va ainsi travailler sur une équation de transport portant sur la densité de probabilité associée à chaque trajectoire. Plus formellement, d'après \cite{Flesch1995}, la formulation classique régissant un modèle de dispersion lagrangien se présente sous la forme d'une équation de Langevin, qui s'écrit: 

\begin{equation}
	\begin{split}
		du_i &= a_i(\Vecx, \vec{\Vecu}, t)dt + b_{i,j}(\Vecx, \vec{\Vecu}, t)d\xi_j  \\
		dx_i &= u_i dt = (\bar{u}_i + U_i)dt
	\end{split}
	\label{eq_langevin}
\end{equation}
où:
 \begin{itemize}
	\item $\Vecx = (x, y, z)$ est la position de la PL  définie par un repère spécifique: $x$ suit l'axe du vent, $y$ suit l'axe perpendiculaire au vent, et $z$ désigne l'élévation verticale classique. 
	\item $\vec{\Vecu}$ est le champ d'écoulement auquel est soumise la PL,
	\item $a_i$ et $b_{i,j}$ sont des fonctions spécifiques de $(\Vecx, \Vecu, t)$ respectivement appelées \textit{drift term} et \textit{random forcing}.
	\item $d\xi_j$ est un incrément aléatoire suivant une distribution gaussienne de moyenne nulle et de variance $dt$.
	\item $\bar{u}_i$ représente le vent moyen et $U_i$ sa composante stochastique.

\end{itemize}

L'expression des fonctions $a_i$ et $b_{i,j}$ varie selon les hypothèses que l'on se fixe sur la nature de la turbulence: une présentation plus détaillée de leur calcul est disponible dans \cite{Wilson1996}. Une fois que ceux-ci sont définis, l'équation \eqref{eq_langevin} est discrétisée et sa résolution permet de calculer un ensemble de trajectoires de PL émanant d'une source dont les paramètres sont connus. Les concentrations volumiques moyennes simulées sont alors obtenues par la somme des particules présentes dans un volume élémentaire $d\Vecx$ autour du point d'observation $\Vecx$ durant un certain temps de résidence. En d'autres termes, en reprenant la définition énoncée dans \cite{Flesch1995}, on peut écrire la concentration moyenne au point $\Vecx$ et à l'instant $t$ comme étant : 
\begin{equation}
	C(\Vecx, t) = \int _{-\infty}^{t} \int_{\Omega} S(\Vecx',t')p(\Vecx, t | \Vecx', t')d\Vecx'dt'
	\label{eq_c_moyen_lagrangien}
\end{equation}
où $\Omega$ est le volume défini par le domaine d'étude, $S(\Vecx',t')$ est la distribution de la source, et $p(\Vecx, t | \Vecx', t')$ est la densité de probabilité sur la position $\Vecx$ et l'instant $t$ des PL de position initiale $\Vecx'$ à l'instant $t'$. \\

\begin{figure}
	\centering
	\includegraphics[width=0.65\textwidth]{lagrangian_ok}
	\caption{Principe du modèle lagrangien: la concentration en $\Vecx$ s'obtient par la somme des PL (en vert) traversant le volume élémentaire $d\Vecx$ durant un certain temps de résidence.}
	\label{fig_schema_lagrangien}
\end{figure}

Afin de calculer le champ d'écoulement {$\Vecu$} auquel sont soumises les PL, plusieurs méthodes sont disponibles. Il est par exemple possible de résoudre les équations de la mécanique des fluides via un outil de simulation de type CFD, ce qui permet une modélisation fine des phénomènes physiques mis en jeu. Une autre possibilité consiste à avoir recours à une simulation dite \textit{CFD simplifiée}, où le champ de vent est interpolé à partir des mesures d'une ou plusieurs stations météorologiques tout en prenant en compte la topographie du terrain. C'est {ce type d'approche qui est présenté dans le paragraphe suivant.}

\subsection{La chaîne de calcul SWIFT-SPRAY}

\textit{Parallel Micro-SWIFT-SPRAY} (PMSS) est une chaîne de calcul constituée de deux éléments distincts: un outil de CFD simplifiée (SWIFT) et un modèle de dispersion lagrangien stochastique (SPRAY). {PMSS est conçu pour être appliqué} dans des études {micro-météorologiques en milieu bâti (site industriel, milieu urbain). La parallélisation rend possible son utilisation sur des domaines plus grands, à l'exemple du cas AIRCITY}\cite{Moussafir2014} où le modèle a été exécuté sur l'ensemble de la ville de Paris.\\

{
\subsubsection{PSWIFT}
}
Le modèle {PSWIFT} permet de produire des champs de vent 3D en exploitant différents types de données météorologiques sur un même site (profils de vent et de température, stations de mesures, sorties de modèles météorologiques de prévision). Il permet notamment de prendre en compte la topographie du milieu, la présence d'obstacles tels que des bâtiments, l'occupation des sols ou encore l'influence de la stabilité atmosphérique. Son fonctionnement {est illustré par la figure \ref{fig_swift_exemple}, et } peut être résumé en quatre étapes :  \\

\begin{enumerate}
	\item Dans un premier temps, les mesures météorologiques reçues en entrée sont interpolées sur les différents points constituant une version discrétisée du domaine.
	\item Dans un deuxième temps, l'effet des obstacles présents dans le domaine sur l'écoulement {est modélisé} via la création de zones spécifiques dans le voisinage de ces obstacles où le champ de vitesse est calculé {par des relations analytiques.}
	\item La troisième étape consiste à ajuster le champ de vent en appliquant un principe de conservation de la masse.
	\item Enfin, la dernière étape consiste à calculer la turbulence intrinsèque à l'écoulement modélisé.\\
\end{enumerate}

\begin{figure}[h!]
	\centering
	\includegraphics[width=0.75\textwidth]{swift_exemple}
	\caption{Exemple de calcul d'un champ de vent autour d'un obstacle avec SWIFT, avant (à gauche) et après (à droite) l'ajustement du champ}
	\label{fig_swift_exemple}
\end{figure}
En sortie de cet enchaînement de calculs, on obtient un champ de vent 3D qui peut alors directement être exploité par le modèle de dispersion {PSPRAY}.


{
\subsubsection{PSPRAY}
}

{PSPRAY} est un modèle de dispersion lagrangien stochastique dont les principes de base suivent les mécanismes présentés à la section \ref{part_lagrangian}. L'implémentation de {PSPRAY} repose sur le critère dit de \textit{well-mixed condition} permettant de donner une formulation explicite aux termes $a_i$ et $b_{i,j}$ de l'équation \eqref{eq_langevin} , et présenté en détail dans les travaux de \cite{Thomson1987}.

En pratique, plusieurs fonctionnalités supplémentaires sont implémentées dans {PSPRAY}, telles que: \\

\begin{itemize}
	\item le "rebond" des particules sur les obstacles,
	\item le calcul de doses pour les sources radioactives,
	\item la prise en compte des différents types de dépôts (secs ou humides).\\
\end{itemize}

{Ces fonctionnalités permettent de représenter de façon réaliste le panache d'un rejet en présence d'obstacles, comme présenté sur la figure \ref{fig_spray_exemple}.}

\begin{figure}[h!]
	\centering
	\includegraphics[width=0.65\textwidth]{spray_exemple}
	\caption{Exemple de champ de concentration calculé par SPRAY dans un domaine de type urbain}
	\label{fig_spray_exemple}
\end{figure}

La combinaison de {PSWIFT} et {PSPRAY} permet ainsi de calculer un champ de concentration sur le domaine étudié, connaissant les paramètres du terme source qui sont soumis en entrée du modèle {PSPRAY}. Dans le contexte de ce chapitre, la chaîne de calcul PMSS permet de:

\begin{itemize}
	\item générer un jeu d'observations synthétiques en simulant un rejet induit par une source que l'on va chercher à retrouver: pour cela, on calcule les valeurs du champ de concentrations en un nombre fini de points du domaine que nous définirons comme étant les observations fournies par les capteurs,
	\item construire les matrices source-récepteur lors de l'exécution de l'algorithme AMIS, nécessaires au processus d'estimation du terme source.\\
\end{itemize}

\subsection{Dualité {direct-rétrograde}}

L'optimisation du calcul des matrices source-récepteur suivant le modèle de l'équation \eqref{eq_AE_4} est un point important: en effet dans une approche directe telle que présentée dans le chapitre précédent, {chaque construction de matrice source-récepteur nécessite une exécution complète du calcul de dispersion}, ce qui rend l'opération d'estimation du terme source très coûteuse en temps de calcul.\\

Nous proposons dans la suite de ce chapitre une amélioration du processus d'estimation privilégiant le calcul des matrices source-récepteur par une approche de type {rétrograde}. Cette méthodologie a initialement été introduite dans \cite{Keats2007} pour ensuite être appliquée sur un algorithme d'estimation de type MCMC, {nous en reprenons les notations pour en rappeler les bases ci-après.}\\

{
On se place dans le domaine spatio-temporel $\Omega \times [0,T]$, et on considère une source $Q$ située au point $\PosSource$ de l'espace, de débit massique constant $q_s$, et de temps d'activation et d'arrêt respectifs $t_{on}$ et $t_{off}$ : 

\begin{equation}
	Q(\Vecx, t) = q_s \delta(\Vecx - \PosSource)\big(H(t-t_{on}) - H(t-t_{off})\big)
\end{equation}
où:
\begin{itemize}
	\item $\delta$ est la distribution de Dirac définie par : 
	\begin{equation}
	\delta(x) = 
	\begin{cases}
	+\infty, & \text{si} ~ x=0 \\
	0, & \text{si} ~ x\neq 0
	\end{cases}
	\end{equation}
	\item $H$ est la fonction de Heaviside définie par:
	\begin{equation}
	H(x) = 
	\begin{cases}
		0, & \text{si} ~ x<0 \\
		1, & \text{si} ~ x \geq 0
	\end{cases}
	\end{equation}
\end{itemize}

On note $C$ le champ de concentration moyen induit par cette source. Soit $R_i$ la concentration mesurée au $i$-ème capteur du réseau considéré: c'est une valeur qui dépend à la fois du champ $C$ et de la "fonction de réponse" $h$ du capteur, cette dernière étant définie par : 
\begin{equation}
	h = h(\Vecx - \Vecx_{R_i}, t - t_{R_i})
\end{equation}

pour un capteur renvoyant la concentration au point $\Vecx_{R_i}$ à l'instant $t_{R_i}$. La valeur de $R_i$ s'obtient alors avec la relation suivante:
\begin{equation}
	\begin{split}
	R_i &= \int_0^T \int_{\Omega} C(\Vecx,t)h(\Vecx - \Vecx_{R_i}, t - t_{R_i})dtd\Vecx \\
	&= \langle C, h \rangle
	\end{split}
\end{equation}

\cite{Keats2007} stipule l'existence d'une relation de dualité entre $C$ et le champ de rétro-concentrations $C^*$, qui résulte de l'expression suivante de $R_i$ en fonction de $C^*$ : 

\begin{equation}
	\begin{split}
		R_i &= \int_0^T \int_{\Omega} Q(\Vecx,t)C^*(\Vecx, t)dtd\Vecx \\
		&= \langle Q, C^* \rangle
	\end{split}
\end{equation}

La relation de dualité direct-rétrograde peut alors être écrite sous la forme suivante: 

\begin{equation}
	\langle C,h \rangle = \langle Q, C^* \rangle
\end{equation}

En pratique, $C^*$ est simulé par un modèle de dispersion rétrograde. Cette relation de dualité permet de traiter le cas particulier où:
\begin{itemize}
	\item sous l'hypothèse d'un "capteur idéal", la fonction de réponse $h$ devient une distribution de Dirac centrée sur le point $(\Vecx, t)$ : 
	\begin{equation}
	\begin{split}
		\langle C,h \rangle &= \int_0^T \int_{\Omega} C(\Vecx,t)\delta(\Vecx - \Vecx_{R_i}, t - t_{R_i})dtd\Vecx \\
		& = C(\Vecx_{R_i}, t_{R_i})
		\end{split}
	\end{equation}
	\item si $Q$ est une source instantanée dont le débit de rejet est unitaire (comme c'est le cas dans l'algorithme AMIS pour chaque particule dont on construit la matrice source-récepteur associée), alors elle est assimilable à une impulsion de type Dirac, avec une masse égale à 1 et centrée sur $(\PosSource, t_s)$ où $t_s$ est l'instant d'émission. Autrement dit:
	\begin{equation}
	\begin{split}
		\langle Q, C^* \rangle &= \int_0^T \int_{\Omega} \delta(\Vecx - \PosSource, t-t_s)C^*(\Vecx, t) dt d\Vecx \\
		&= C^*(\PosSource, t_s)
		\end{split}
	\end{equation}
\end{itemize}

Sous ces hypothèse, on obtient:
\begin{equation}
C(\Vecx_{R_i}, t_{R_i}) = C^* (\PosSource, t_s)
\end{equation}


\subsection{Intégration d'un modèle {rétrograde} au processus d'estimation}

\subsubsection{Utilisation de RetroSPRAY dans l'AMIS}

{PMSS contient une implémentation rétrograde du modèle SPRAY, appelée Retro-SPRAY, qui procède à la résolution en mode inverse des équations de Langevin (le détail des calculs est développé dans \cite{Flesch1995} et \cite{Wilson2009}).} \\

Si on examine plus en détail la construction de la matrice source-récepteur, on peut réécrire l'équation \eqref{eq_AE_4} sous une forme plus explicite: notons $C(R_i,t_j |\VecTheta, t'_n)$ la concentration moyenne au capteur $R_i$ à l'instant $t_j$ résultant d'une source située à la position $\VecTheta$ et ayant émis un rejet unitaire à l'instant $t'_n$. La version directe  de la matrice source-récepteur pour $\VecTheta$ s'écrit:

\begin{equation}
	\MatC^f(\VecTheta) = 
	\begin{pmatrix}
		C(R_1,t_1 | \VecTheta, t'_1) & C(R_1,t_1 | \VecTheta, t'_2) & \cdots & C(R_1, t_1 |\VecTheta, t'_{T_s}) \\
		C(R_1,t_2 | \VecTheta, t'_1) & C(R_1,t_2 | \VecTheta, t'_2) & \cdots & C(R_1, t_2 |\VecTheta, t'_{T_s}) \\
		\vdots & \vdots & & \vdots \\
		C(R_1,t_{T_c} | \VecTheta, t'_1) & C(R_1,t_{T_c} | \VecTheta, t'_2) & \cdots & C(R_1, t_{T_c} |\VecTheta, t'_{T_s}) \\
		C(R_2,t_1| \VecTheta, t'_1) & C(R_2,t_1 | \VecTheta, t'_2) & \cdots & C(R_2, t_1 |\VecTheta, t'_{T_s}) \\
		\vdots & \vdots & & \vdots \\
		\vdots & \vdots & & \vdots \\
		C(R_{N_c},t_{T_c} | \VecTheta, t'_1) & C(R_{N_c},t_{T_c} | \VecTheta, t'_2) & \cdots & C(R_{N_c}, t_{T_c} |\VecTheta, t'_{T_s}) \\
	\end{pmatrix}
	\label{eq_matrix_forward}
\end{equation}

En appliquant la relation de dualité {direct-rétrograde}, dans l'espace dual où le modèle {rétrograde} opère, les sources deviennent des "rétro-capteurs", et les capteurs deviennent des "rétro-sources". On définit alors $C^*(\VecTheta,t'_n | R_i, t_j)$ comme la rétro-concentration mesurée au point $\VecTheta$ à l'instant $t'_n$ provenant d'une rétro-source située à la position $R_i$ et ayant émis un rétro-rejet unitaire à l'instant $t_j$. La version {rétrograde} de $\MatC^f$ s'écrit alors:

\begin{equation}
	\MatC^b (\VecTheta)= 
	\begin{pmatrix}
		C^*(\VecTheta, t'_1 | R_1, t_1) & C^*(\VecTheta, t'_2 | R_1, t_1) & \cdots & C^*(\VecTheta, t'_{T_s} | R_1, t_1) \\
		C^*(\VecTheta, t'_1 | R_1, t_2) & C^*(\VecTheta, t'_2 | R_1, t_2) & \cdots & C^*(\VecTheta, t'_{T_s} | R_1, t_1) \\
		\vdots & \vdots & & \vdots \\
		C^*(\VecTheta, t'_1| R_1,t_{T_c}) & C^*(\VecTheta, t'_2 | R_1, t_{T_c}) & \cdots & C^*(\VecTheta, t'_{T_s} | R_1, t_{T_c}) \\ 
		\vdots & \vdots & & \vdots \\
		\vdots & \vdots & & \vdots \\
		C^*(\VecTheta, t'_1| R_{N_c},t_{T_c}) & C^*(\VecTheta, t'_2 | R_{N_c}, t_{T_c}) & \cdots & C^*(\VecTheta, t'_{T_s} | R_{N_c}, t_{T_c}) \\ 
	\end{pmatrix}
	\label{eq_matrix_backward}
\end{equation}

Dans l'algorithme AMIS, au moment de calculer la vraisemblance de chaque particule échantillonnée, on peut ainsi faire désormais intervenir $\MatC^b$ comme matrice source-récepteur.

\subsubsection{Avantages}

{L'utilisation de l'approche rétrograde} permet de n'avoir à faire qu'un seul calcul de rétro-dispersion par capteur, qui donne alors l'ensemble des valeurs de rétro-concentrations sur le domaine. {Ces opérations sont faites en amont du schéma itératif de l'AMIS: le domaine est divisé en plusieurs mailles sur lesquelles sont pré-calculées valeurs de $C^*$. A partir de ces résultats, une série de fichiers binaires est créée puis stockée sur le disque. Ainsi, au moment de l'estimation du terme source, les valeurs à insérer dans les matrices source-récepteur sont déjà disponibles: il suffit d'associer les positions des particules tirées aux centres des mailles les plus proches sur lesquelles les valeurs de $C^*$ ont été obtenues. Pour cela, il suffit alors d'effectuer une opération de lecture sur les fichiers binaires contenant les rétro-concentrations, ce qui permet ainsi de ne plus avoir à lancer le modèle de dispersion à chaque nouvelle particule tirée lors de la phase d'échantillonnage de l'AMIS.\\}

Même si cette opération n'est plus directement intégrée à l'algorithme d'estimation, elle demande néanmoins une certaine quantité de calculs. Cette dernière peut toutefois être réduite si on choisit de paralléliser l'opération de génération des fichiers binaires de rétro-concentration. Comme il s'agit de tâches parfaitement indépendantes les unes par rapport aux autres, la parallélisation est facilement réalisable (\textit{embarrasingly parallel jobs}). Si on doit générer $n_b$ fichiers binaires en allouant $n_p$ coeurs de calcul par opération de génération, cela requiert la disponibilité de $n_b \times n_p$ coeurs. Si une telle quantité de ressources n'est pas immédiatement disponible, il peut être plus judicieux de créer séquentiellement ces fichiers binaires. L'approche séquentielle est également à privilégier si on ne dispose pas d'une architecture de type \textit{cluster} permettant une parallélisation massive. 
\newpage
{
\section{Exemple d'application sur un terrain rural}
}

Dans cette section, nous présentons un première application sur une situation  simple dans un contexte non-urbain. Il s'agit d'un exemple synthétique dont les données et les paramètres sont issus d'une simulation ayant permis la validation du modèle {Retro-SPRAY}.

\subsection{Présentation du cas-test}

Nous considérons ici un cas-test en milieu rural reproduisant une émission accidentelle depuis un site industriel dans une zone située près de la commune de Beaune, en Bourgogne, en présence d'une topographie réelle {vallonnée} (figure \ref{fig_beaune_relief}). 

\begin{figure}[h!]
	\centering
	\includegraphics[width=0.8\textwidth]{beaune_relief_capteurs}
	\caption{Superposition du relief, de l'emplacement des capteurs et de la source du cas-test Beaune}
	\label{fig_beaune_relief}
\end{figure}

\subsubsection{Caractéristiques du domaine}
Le domaine considéré couvre une surface de \SI{6}{\square\kilo\meter} avec une source unique et un réseau relativement dense de 25 capteurs disposés en quinconce et couvrant toute la superficie du domaine (figure \ref{fig_beaune_relief}).

Pour la simulation, le domaine est discrétisé en une grille de $300 \times 300$ mailles, avec une résolution du maillage en $x$ et $y$ de \SI{20}{\meter}.

\subsubsection{Paramètres météorologiques}
Sur toute la durée de la simulation, nous considérons un vent constant en vitesse (\SI{1.5}{\m\per\second}) et en direction ($330\degres$). Le champ de vent produit par SWIFT est ainsi relativement homogène, comme observé en figure \ref{fig_beaune_vent}.

\begin{figure}[h!]
	\centering
	\includegraphics[width=0.8\textwidth]{beaune_vent}
	\caption{Champ de vent calculé par SWIFT pour le cas-test Beaune}
	\label{fig_beaune_vent}
\end{figure}


\subsubsection{Capteurs, source et simulation des observations}
On considère un réseau de 25 capteurs disposés de façon à couvrir tout le domaine, et placés à une hauteur de \SI{10}{\meter}, égale à celle de la source. Chaque capteur fournit des observations de concentrations moyennes entre 10h05 et 12h00, avec une plage de moyennage de 5 minutes.

Concernant la source, celle-ci est placée dans la partie nord (voir figure \ref{fig_beaune_relief}) afin que le panache résultant couvre une partie suffisante du domaine. Elle est située à une altitude de \SI{10}{\m}, identique à celle des capteurs: {le processus d'estimation est limité à} la reconstruction de la position aux coordonnées $(x,y)$. La source émet un rejet unique entre 10h15 et 11h, avec un débit constant de 1850 unités/s. {Lors de la simulation des observations, elle est modélisée par un volume de (\SI{15}{\meter}$\times$\SI{15}{\meter}$\times$\SI{10}{\meter}), les valeurs obtenues sont représentées en figure \ref{fig_observations_25CAPTEURS}.}\\

\begin{figure}[h!]
	\centering
	\includegraphics[width=0.8\textwidth]{concentrations_beaune.png}
	\caption{Cas-test Beaune: concentrations mesurées aux capteurs}
	\label{fig_observations_25CAPTEURS}
\end{figure}

{
Il est à noter que pour générer les observations de la figure \ref{fig_observations_25CAPTEURS}, un \textit{run} spécifique du modèle de dispersion a été exécuté, sans réutiliser les valeurs pré-calculées présentes dans les fichiers binaires. Cela entraîne une marge d'incertitude sur les mesures, qu'il est possible d'associer à l'introduction d'une erreur de modèle non-nulle.
}

\subsubsection{Premiers résultats}

Un \textit{run} préliminaire de l'AMIS servant de \textit{benchmark} a été lancé avec les paramètres suivants:
\begin{itemize}
	\item 10 itérations,
	\item 100 particules générées par itération,
	\item $\varObs = 2\times 10^{-6}$, qui fixe un degré d'incertitude suffisant autour des observations de la figure \ref{fig_observations_25CAPTEURS},
	\item une discrétisation temporelle de la source par paliers de 5 minutes, avec les paramètres a priori $\VecMeanQ = (0,\cdots,0)$ et $\varQ = 2\times 10^6$.
\end{itemize}

Avec ces paramètres, l'algorithme d'estimation parvient à fournir une reconstruction relativement correcte des paramètres, aussi bien concernant la localisation de la source (figures \ref{fig_25C_prelim_X} et \ref{fig_25C_prelim_Y}) que la reconstruction de son profil d'émission (figure \ref{fig_25C_prelim_Q}).\\

\begin{figure}[h!]
	\centering
	\begin{subfigure}[t]{0.5\textwidth}
		\centering
		\includegraphics[width=1\textwidth]{beaune_25C_prelim_x}
		\caption{Position en $x$}
		\label{fig_25C_prelim_X}
	\end{subfigure}%
	\begin{subfigure}[t]{0.5\textwidth}
		\centering
		\includegraphics[width=1\textwidth]{beaune_25C_prelim_y}
		\caption{Position en $y$}
		\label{fig_25C_prelim_Y}
	\end{subfigure}
	\begin{subfigure}[t]{0.65\textwidth}
		\centering
		\includegraphics[width=1\textwidth]{beaune_25C_prelim_q}
		\caption{Profil d'émission avec intervalle de confiance à  $\pm 2 \widetilde{\sigma}_q^2$ (gris)}
		\label{fig_25C_prelim_Q}
	\end{subfigure} 
	\caption{Résultats du \textit{benchmark} de l'algorithme d'estimation sur le cas-test Beaune}
	\label{fig_25C_prelim}
\end{figure}

Pour mesurer l'erreur relative d'une estimation ponctuelle par rapport à la vraie position de la source, on utilise la métrique suivante:

\begin{equation}
	 r_d = \dfrac{d(\Vecx_s,\widehat{\Vecx}_{MMSE})}{L_{\mathcal{D}}}
\end{equation}
où:
\begin{itemize}
	\item $L_{\mathcal{D}}$ est la diagonale du domaine $\mathcal{D}$ considéré, autrement dit la plus grande distance possible entre deux points: cela permet de borner $r_d$ entre 0 et 1,
	\item $\Vecx_s$ est le centre de la maille où est située la source,
	\item $\widehat{\Vecx} _{MMSE}$ est l'estimation ponctuelle par MMSE de la position de la source obtenue via les particules et les poids d'importance fournis en sortie de l'AMIS.
\end{itemize}

\NdFS{Pour mesurer l'erreur d'estimation du débit, on utilise la même métrique (MSE) que dans le chapitre précédent (\eqref{eq_AE_31}).}

Dans l'exemple de la figure \ref{fig_25C_prelim}, on obtient ainsi une erreur relative de $r_d = 0.017$ pour la position, et \NdFS{une erreur d'estimation du profil d'émission valant $\text{MSE}(\widehat{\VecQSource}, \VecQSource) = 296.909$.}\\

Nous étudions dans les paragraphes suivants l'influence des paramètres de variance d'observation $\varObs$ et de variance a priori $\varQ$ du profil d'émission $\VecQSource$ sur la qualité de l'estimation.

\newpage

 Une telle étude paramétrique permet en effet de donner du sens à ces paramètres, ceux-ci devant être spécifiés par l'utilisateur parmi les variables d'entrée de l'algorithme d'estimation. 

\subsection{Influence de la variance d'observation}

La variance d'observation $\varObs$ est le paramètre qui reflète la "confiance" donnée aux observations $\VecObs$. Comme expliqué au Chapitre 3, elle caractérise l'ensemble des erreurs à l'origine de l'écart entre les valeurs issues du modèle de données et la réalité physique. Comme il a été expliqué précédemment, l'erreur d'observation est une agrégation de diverses sources individuelles d'erreur (modèle, instrumentation...). Sans chercher à mener une analyse approfondie sur la quantification des incertitudes autour des observations, nous cherchons plutôt ici à avoir une vision d'ensemble de l'influence du paramètre $\varObs$ sur la qualité de l'estimation du terme source. Pour cela, on garde les mêmes paramètres que le \textit{benchmark} de la figure \ref{fig_25C_prelim}, on choisit ensuite une plage de valeurs de $\varObs$ à tester, et pour chacune de ces valeurs, on lance 100 \textit{runs} de l'AMIS. Ce dernier faisant intervenir des tirages aléatoires, le fait de considérer les résultats issus d'un nombre suffisant de \textit{runs} permet de voir si la qualité des estimations n'est pas perturbée par ces aspects stochastiques. 

Concernant la localisation de la source, on observe sur les figures \ref{fig_25C_analyse_varobs_x} et \ref{fig_25C_analyse_varobs_y} que les résultats sont réguliers: l'estimation de la position est bonne pour les valeurs inférieures à $5\times 10^{-6}$ puis se dégrade progressivement jusqu'à {ce qu'une zone précise de l'espace devienne difficile à définir} (\ref{varF_x} et \ref{varF_y}).

Pour le profil d'estimation {du débit de rejet} sur la figure \ref{fig_25C_analyse_varobs_q}, celui-ci est relativement bien estimé pour $\varObs=10^{-7}$, les variations étant de plus en plus importantes au fur et à mesure que la valeur de la variance d'observation augmente. Les courbes d'erreur de la figure \ref{fig_25C_varobs_erreurs} donnent un aperçu de l'influence générale de la variance d'observation.

\begin{figure}[h!]
	\centering
	\includegraphics[width=0.8\textwidth]{25C_varobs_errors.png}
	\caption{Courbes d'erreur pour l'analyse paramétrique de $\varObs$}
	\label{fig_25C_varobs_erreurs}
\end{figure}

Certaines des valeurs estimées pour le débit {sont} négatives, or il a été constaté que l'application de la contrainte de positivité a tendance à {introduire un biais sur les valeurs de $\tilde{\VecMeanQ}$, comme le montrent les figures \ref{fig_AE_2} et \ref{fig_sansavecPC}.} Il a donc été décidé de ne pas implémenter cette contrainte, car malgré la présence de quelques valeurs négatives, le sens physique des résultats d'estimation est globalement respecté.

\begin{figure}[h!]
	\centering
	\begin{subfigure}[t]{0.5\textwidth}
		\centering
		\includegraphics[width=1\textwidth]{1part_sansPC.png}
		\caption{}
		\label{sansPC}
	\end{subfigure}%
	\begin{subfigure}[t]{0.5\textwidth}
		\centering
		\includegraphics[width=1\textwidth]{1part_avecPC.png}
		\caption{}
		\label{avecPC}
	\end{subfigure}
	\caption{Comparaison de l'estimation de $\tilde{\VecMeanQ}$ sur une particule dans la maille de la source sans (à gauche) et avec (à droite) la contrainte de positivité}
	\label{fig_sansavecPC}
\end{figure}

En examinant ces résultats, on pourrait considérer la valeur $\varObs = 10^{-7}$ comme étant celle qui donne les meilleurs résultats. Il faut cependant rappeler que dans notre cas d'étude, nous utilisons des observations synthétiques non-bruitées, ce qui permet d'accorder une plus grande confiance à ces observations et donc de réduire la valeur de $\varObs$. On ne peut pas forcément en faire autant si des sources d'incertitudes supplémentaires entrent en jeu, par exemple avec des valeurs de concentrations expérimentales issues de mesures réelles, car cela reviendrait à diminuer la marge d'incertitude sur les mesures alors que le caractère aléatoire de ces dernières est plus accentué. 

\begin{figure}[p!]
	\centering
	\begin{subfigure}[t]{0.5\textwidth}
		\centering
		\includegraphics[width=1\textwidth]{25C_varobs_X_A.png}
		\caption{}
		\label{varA_x}
	\end{subfigure}%
	\begin{subfigure}[t]{0.5\textwidth}
		\centering
		\includegraphics[width=1\textwidth]{25C_varobs_X_B.png}
		\caption{}
		\label{varB_x}
	\end{subfigure}
	\begin{subfigure}[t]{0.5\textwidth}
		\centering
		\includegraphics[width=1\textwidth]{25C_varobs_X_C.png}
		\caption{}
		\label{varC_x}
	\end{subfigure}%
	\begin{subfigure}[t]{0.5\textwidth}
		\centering
		\includegraphics[width=1\textwidth]{25C_varobs_X_D.png}
		\caption{}
		\label{varD_x}
	\end{subfigure}
	\begin{subfigure}[t]{0.5\textwidth}
		\centering
		\includegraphics[width=1\textwidth]{25C_varobs_X_E.png}
		\caption{}
		\label{varE_x}
	\end{subfigure}%
	\begin{subfigure}[t]{0.5\textwidth}
		\centering
		\includegraphics[width=1\textwidth]{25C_varobs_X_F.png}
		\caption{}
		\label{varF_x}
	\end{subfigure}
	\caption{Analyse paramétrique sur la variance d'observation $\varObs$ pour le cas-test Beaune (25 capteurs): localisation en $x$ de la source}
	\label{fig_25C_analyse_varobs_x}
	
\end{figure}

\begin{figure}[p!]
	\centering
	\begin{subfigure}[t]{0.5\textwidth}
		\centering
		\includegraphics[width=1\textwidth]{25C_varobs_Y_A.png}
		\caption{}
		\label{varA_y}
	\end{subfigure}%
	\begin{subfigure}[t]{0.5\textwidth}
		\centering
		\includegraphics[width=1\textwidth]{25C_varobs_Y_B.png}
		\caption{}
		\label{varB_y}
	\end{subfigure}
	\begin{subfigure}[t]{0.5\textwidth}
		\centering
		\includegraphics[width=1\textwidth]{25C_varobs_Y_C.png}
		\caption{}
		\label{varC_y}
	\end{subfigure}%
	\begin{subfigure}[t]{0.5\textwidth}
		\centering
		\includegraphics[width=1\textwidth]{25C_varobs_Y_D.png}
		\caption{}
		\label{varD_y}
	\end{subfigure}
	\begin{subfigure}[t]{0.5\textwidth}
		\centering
		\includegraphics[width=1\textwidth]{25C_varobs_Y_E.png}
		\caption{}
		\label{varE_y}
	\end{subfigure}%
	\begin{subfigure}[t]{0.5\textwidth}
		\centering
		\includegraphics[width=1\textwidth]{25C_varobs_Y_F.png}
		\caption{}
		\label{varF_y}
	\end{subfigure}
	\caption{Analyse paramétrique sur la variance d'observation $\varObs$ pour le cas-test Beaune (25 capteurs): localisation en $y$ de la source}
	\label{fig_25C_analyse_varobs_y}
	
\end{figure}


\begin{figure}[p!]
	\centering
	\begin{subfigure}[t]{0.5\textwidth}
		\centering
		\includegraphics[width=1\textwidth]{25C_varobs_Q_A.png}
		\caption{}
		\label{varA_q}
	\end{subfigure}%
	\begin{subfigure}[t]{0.5\textwidth}
		\centering
		\includegraphics[width=1\textwidth]{25C_varobs_Q_B.png}
		\caption{}
		\label{varB_q}
	\end{subfigure}
	\begin{subfigure}[t]{0.5\textwidth}
		\centering
		\includegraphics[width=1\textwidth]{25C_varobs_Q_C.png}
		\caption{}
		\label{varC_q}
	\end{subfigure}%
	\begin{subfigure}[t]{0.5\textwidth}
		\centering
		\includegraphics[width=1\textwidth]{25C_varobs_Q_D.png}
		\caption{}
		\label{varD_q}
	\end{subfigure}
	\begin{subfigure}[t]{0.5\textwidth}
		\centering
		\includegraphics[width=1\textwidth]{25C_varobs_Q_E.png}
		\caption{}
		\label{varE_q}
	\end{subfigure}%
	\begin{subfigure}[t]{0.5\textwidth}
		\centering
		\includegraphics[width=1\textwidth]{25C_varobs_Q_F.png}
		\caption{}
		\label{varF_q}
	\end{subfigure}
	\caption{Analyse paramétrique sur la variance d'observation $\varObs$ pour le cas-test Beaune (25 capteurs): reconstruction du profil d'émission}
	\label{fig_25C_analyse_varobs_q}
\end{figure}


\subsection{Influence de la variance a priori du profil d'émission}

La variance a priori $\varQ$ peut être vue comme une hypothèse de départ sur l'amplitude possible des valeurs du profil d'émission. Son influence s'applique à la fois sur la qualité de la reconstruction de $\VecQSource$, mais également sur celle de la localisation de la source, car $\varQ$ intervient dans l'algorithme AMIS lors du calcul de la vraisemblance des particules. 

Pour cette analyse, on utilise le même procédé que pour l'étude de $\varObs$, en reprenant les paramètres du \textit{benchmark} et en exécutant 100 \textit{runs} de l'AMIS {sur différentes valeurs de $\varQ$.}

Pour l'estimation de la position, on constate que plus la valeur de $\varQ$ considérée est grande, plus la source aura tendance à être estimée en amont  par rapport à la direction du vent. En effet, pour la même position potentielle de la source, des valeurs plus élevées pour les quantités de polluant rejetées impliquent des mesures de concentration plus importantes aux capteurs: l'algorithme va donc ajuster cette position en remontant l'axe du vent pour avoir une meilleure vraisemblance par rapport aux observations. Ce phénomène de décalage spatial est bien visible sur la figure \ref{fig_25C_varQ_boxplots}, où les \textit{boxplots} des estimations ponctuelles sont tracées pour chacune des valeurs de $\varQ$ testées.

\begin{figure}[h!]
	\centering
	\begin{subfigure}[t]{0.5\textwidth}
		\centering
		\includegraphics[width=1\textwidth]{25C_varQ_boxplot_x.png}
		\caption{}
		\label{varQ_boxplot_x}
	\end{subfigure}%         	
	\begin{subfigure}[t]{0.5\textwidth}
		\centering
		\includegraphics[width=1\textwidth]{25C_varQ_boxplot_y.png}
		\caption{}
		\label{varQ_boxplot_y}
	\end{subfigure}%
	\caption{\textit{Boxplots} des estimations par MMSE de la position de la source sur 100 runs, comparaison avec les valeurs réelles (en pointillés noirs)}
	\label{fig_25C_varQ_boxplots}
\end{figure}

La meilleure estimation du profil d'émission est obtenue pour $\varQ=2\times 10^6$ (voir figure \ref{fig_25C_analyse_varq_q}, ce qui coïncide avec la meilleure position estimée dans la figure \ref{fig_25C_varQ_boxplots}. Pour des valeurs plus faibles de $\varQ$ le débit est sous-estimé, à l'inverse pour des valeurs plus élevées le débit est surestimé. On peut résumer ce comportement grâce aux courbes d'erreur de la figure \ref{fig_25C_varq_erreurs}.

\begin{figure}[h!]
	\centering
	\includegraphics[width=0.8\textwidth]{25C_varQ_errors.png}
	\caption{Courbes d'erreur pour l'analyse paramétrique de $\varQ$}
	\label{fig_25C_varq_erreurs}
\end{figure}

\begin{figure}[p!]
	\centering
	\begin{subfigure}[t]{0.5\textwidth}
		\centering
		\includegraphics[width=1\textwidth]{25C_varq_X_A.png}
		\caption{}
		\label{varq_A_x}
	\end{subfigure}%
	\begin{subfigure}[t]{0.5\textwidth}
		\centering
		\includegraphics[width=1\textwidth]{25C_varq_X_B.png}
		\caption{}
		\label{varq_B_x}
	\end{subfigure}
	\begin{subfigure}[t]{0.5\textwidth}
		\centering
		\includegraphics[width=1\textwidth]{25C_varq_X_C.png}
		\caption{}
		\label{varq_C_x}
	\end{subfigure}%
	\begin{subfigure}[t]{0.5\textwidth}
		\centering
		\includegraphics[width=1\textwidth]{25C_varq_X_D.png}
		\caption{}
		\label{varq_D_x}
	\end{subfigure}
	\begin{subfigure}[t]{0.5\textwidth}
		\centering
		\includegraphics[width=1\textwidth]{25C_varq_X_E.png}
		\caption{}
		\label{varq_E_x}
	\end{subfigure}%
	\begin{subfigure}[t]{0.5\textwidth}
		\centering
		\includegraphics[width=1\textwidth]{25C_varq_X_F.png}
		\caption{}
		\label{varq_F_x}
	\end{subfigure}
	\caption{Analyse paramétrique sur la variance a priori $\varQ$ pour le cas-test Beaune (25 capteurs): localisation en $x$ de la source}
	\label{fig_25C_analyse_varq_x}
\end{figure}

\begin{figure}[p!]
	\centering
	\begin{subfigure}[t]{0.5\textwidth}
		\centering
		\includegraphics[width=1\textwidth]{25C_varQ_Y_A.png}
		\caption{}
		\label{varq_A_y}
	\end{subfigure}%
	\begin{subfigure}[t]{0.5\textwidth}
		\centering
		\includegraphics[width=1\textwidth]{25C_varQ_Y_B.png}
		\caption{}
		\label{varq_B_y}
	\end{subfigure}
	\begin{subfigure}[t]{0.5\textwidth}
		\centering
		\includegraphics[width=1\textwidth]{25C_varQ_Y_C.png}
		\caption{}
		\label{varq_C_y}
	\end{subfigure}%
	\begin{subfigure}[t]{0.5\textwidth}
		\centering
		\includegraphics[width=1\textwidth]{25C_varQ_Y_D.png}
		\caption{}
		\label{varq_D_y}
	\end{subfigure}
	\begin{subfigure}[t]{0.5\textwidth}
		\centering
		\includegraphics[width=1\textwidth]{25C_varQ_Y_E.png}
		\caption{}
		\label{varq_E_y}
	\end{subfigure}%
	\begin{subfigure}[t]{0.5\textwidth}
		\centering
		\includegraphics[width=1\textwidth]{25C_varQ_Y_F.png}
		\caption{}
		\label{varq_F_y}
	\end{subfigure}
	\caption{Analyse paramétrique sur la variance a priori $\varQ$ pour le cas-test Beaune (25 capteurs): localisation en $y$ de la source}
	\label{fig_25C_analyse_varq_y}
\end{figure}

\begin{figure}[p!]
	\centering
	\begin{subfigure}[t]{0.5\textwidth}
		\centering
		\includegraphics[width=1\textwidth]{25C_varQ_Q_A.png}
		\caption{}
		\label{varq_A_q}
	\end{subfigure}%
	\begin{subfigure}[t]{0.5\textwidth}
		\centering
		\includegraphics[width=1\textwidth]{25C_varQ_Q_B.png}
		\caption{}
		\label{varq_B_q}
	\end{subfigure}
	\begin{subfigure}[t]{0.5\textwidth}
		\centering
		\includegraphics[width=1\textwidth]{25C_varQ_Q_C.png}
		\caption{}
		\label{varq_C_q}
	\end{subfigure}%
	\begin{subfigure}[t]{0.5\textwidth}
		\centering
		\includegraphics[width=1\textwidth]{25C_varQ_Q_D.png}
		\caption{}
		\label{varq_D_q}
	\end{subfigure}
	\begin{subfigure}[t]{0.5\textwidth}
		\centering
		\includegraphics[width=1\textwidth]{25C_varQ_Q_E.png}
		\caption{}
		\label{varq_E_q}
	\end{subfigure}%
	\begin{subfigure}[t]{0.5\textwidth}
		\centering
		\includegraphics[width=1\textwidth]{25C_varQ_Q_F.png}
		\caption{}
		\label{varq_F_q}
	\end{subfigure}
	\caption{Analyse paramétrique sur la variance a priori $\varQ$ pour le cas-test Beaune (25 capteurs): reconstruction du profil d'émission}
	\label{fig_25C_analyse_varq_q}
\end{figure}



\subsection{Influence de la densité du réseau de capteurs}

{Le nombre et la répartition des capteurs constituent} un facteur important: 

\begin{itemize}
	\item du point de vue théorique, son étude permet de mieux comprendre le comportement de l'algorithme d'estimation et quantifier son importance par rapport aux autres variables,
	\item en pratique, l'analyse des résultats d'estimation en fonction de la configuration des capteurs permet un meilleur dimensionnement du réseau de mesure.\\
\end{itemize}

\subsubsection{Impact du capteur R8}

Dans un premier temps, nous nous intéressons à l'influence du capteur R8, qui est le récepteur le plus proche de la source à mesurer des concentrations non-nulles. Pour cela, nous le retirons du réseau, et nous effectuons l'opération d'estimation du terme source avec les observations issues des 24 capteurs restants, en suivant les paramètres du \textit{benchmark}. En pratique, cela reviendrait par exemple à simuler la panne du capteur R8 durant la fenêtre temporelle d'observation.\\


\begin{figure}[h!]
	\centering
	\begin{subfigure}[t]{0.5\textwidth}
		\centering
		\includegraphics[width=1\textwidth]{R8_kde_x_compare.png}
		\caption{Position en $x$}
		\label{R8_x}
	\end{subfigure}%
	\begin{subfigure}[t]{0.5\textwidth}
		\centering
		\includegraphics[width=1\textwidth]{R8_kde_y_compare.png}
		\caption{Position en $y$}
		\label{R8_y}
	\end{subfigure}
	\begin{subfigure}[t]{0.65\textwidth}
		\centering
		\includegraphics[width=1\textwidth]{R8_q_est_compare.png}
		\caption{Profil d'émission avec intervalle de confiance à  $\pm 2 \tilde{\sigma}_q^2$}
		\label{R8_q}
	\end{subfigure} 
	\caption{Résultats d'un \textit{run} de l'algorithme d'estimation sans (jaune) et avec (bleu) le capteur R8}
	\label{fig_R8_compare}
\end{figure}

A paramètres identiques, la source est vue plus en amont sur l'axe du vent par rapport à sa position réelle (figures \ref{R8_x} et \ref{R8_y}). De plus, sans le capteur R8, le débit est relativement sous-estimé par rapport aux valeurs attendues, et le rejet reconstitué commence et s'achève plus tôt que prévu (figure \ref{R8_q}). On observe ainsi que les mesures fournies par le capteur R8 apportent une information non-négligeable permettant d'accroître la précision de l'estimation, à la fois sur les aspects spatiaux, temporels, et de quantité émise. {De manière générale, on peut donc en déduire que la présence d'un capteur suffisamment proche de la source permet d'assurer une bonne reconstruction de celle-ci.}

\subsubsection{Impact d'un réseau réduit}

On a vu que le rôle individuel des capteurs peut être important pour pouvoir reconstruire correctement un terme source. Nous cherchons ici à savoir comment varie la qualité de cette reconstruction si l'ensemble du réseau est modifié.

On réduit ainsi la taille de notre réseau à 9 capteurs (figure \ref{fig_reseaux_25_9}), répartis de façon homogène sur le domaine. Cela permet de simuler une situation où le nombre d'instruments de mesure des concentrations est moindre (par exemple, dans le cas où les capteurs sont coûteux à acheter, à déployer {et à maintenir en bon état de fonctionnement}), et également d'apprécier le comportement de l'algorithme d'estimation lorsque la représentativité spatiale des mesures est limitée. Les paramètres d'entrée de l'algorithme d'estimation sont toujours ceux du \textit{benchmark}.\\

\begin{figure}[h!]
	\centering
	\begin{subfigure}[t]{0.5\textwidth}
		\centering
		\includegraphics[width=1\textwidth]{reseau_25C.png}
		\caption{}
		\label{reseau_25C}
	\end{subfigure}%
	\begin{subfigure}[t]{0.5\textwidth}
		\centering
		\includegraphics[width=1\textwidth]{reseau_9C.png}
		\caption{}
		\label{reseau_9C}
	\end{subfigure}
	\caption{Réduction de la densité du réseau de capteurs: passage de 25 (gauche) à 9 (droite) capteurs}
	\label{fig_reseaux_25_9}
\end{figure}


\begin{figure}[h!]
	\centering
	\begin{subfigure}[t]{0.5\textwidth}
		\centering
		\includegraphics[width=1\textwidth]{kde_x_compare_all.png}
		\caption{}
		\label{kde_x_all}
	\end{subfigure}%
	\begin{subfigure}[t]{0.5\textwidth}
		\centering
		\includegraphics[width=1\textwidth]{kde_y_compare_all.png}
		\caption{}
		\label{kde_y_all}
	\end{subfigure}
	\begin{subfigure}[t]{0.33\textwidth}
		\centering
		\includegraphics[width=1\textwidth]{q_9C.png}
		\caption{}
		\label{q_9C}
	\end{subfigure}%
	\begin{subfigure}[t]{0.33\textwidth}
		\centering
		\includegraphics[width=1\textwidth]{q_24C.png}
		\caption{}
		\label{q_24C}
	\end{subfigure}%
	\begin{subfigure}[t]{0.33\textwidth}
		\centering
		\includegraphics[width=1\textwidth]{q_25C.png}
		\caption{}
		\label{q_25C}
	\end{subfigure}
	\caption{Résultats d'un \textit{run} de l'algorithme d'estimation sans (jaune) et avec (bleu) le capteur R8, comparaison avec un réseau réduit (vert)}
	\label{}
\end{figure}



\newpage Les résultats de  localisation de la source montrent une dégradation de l'estimation de la position similaire au cas où le capteur R8 n'est pas présent (figures \ref{kde_x_all} et \ref{kde_y_all}). Il en va de même pour pour l'estimation du profil de rejet (figures \ref{q_9C}, \ref{q_24C} et \ref{q_25C}), où la marge d'incertitude est même légèrement moins grande dans la configuration à capteurs. 

On peut ainsi en déduire que les capteurs n'observant aucune concentration  non-nulle et qui ont été retirés pour établir la configuration à 9 capteurs ont bien un rôle informatif utile au processus d'estimation, et leur apport qualitatif par rapport à la présence du capteur R8 dans le réseau à 25 capteurs semble de même importance.\\

De façon générale, on peut ainsi voir que l'efficacité de l'estimation du terme source repose sur un certain nombre de paramètres d'entrée pour l'algorithme AMIS. Ceux-ci peuvent dépendre entièrement de la configuration du cas-test (nombre de capteurs et disposition sur le domaine), ou être définis {arbitrairement} par l'utilisateur. Dans ce dernier cas, on a constaté qu'il est difficile d'établir des critères empiriques pour choisir des valeurs par défaut à affecter aux paramètres de variance $\varObs$ et $\varQ$. Pour la variance d'observation, un choix qualitatif a été fait par rapport à l'allure des mesures de la figure \ref{fig_observations_25CAPTEURS}, mais dans d'autres situations pratiques, l'évaluation de $\varObs$ peut se révéler plus compliquée:
\begin{itemize}
	\item si on sait d'avance que certains capteurs du réseau sont plus fiables que d'autres et qu'on choisit d'accorder une plus grande confiance à ces capteurs, alors l'hypothèse d'une loi d'incertitude identiquement distribuée sur les éléments du vecteur d'observation n'est plus valable,
	\item dans un cas expérimental où on aurait choisi une variance d'observation trop faible,{la plage de probabilité non-nulle sur la densité a posteriori estimée peut être réduite au point de ne plus inclure la vraie valeur du paramètre recherché.}
\end{itemize}

Une alternative plausible à cette recherche difficile des paramètres initiaux consisterait à travailler sur un ensemble de scénarios possibles, et ainsi d'étudier plusieurs possibilités quant aux valeurs des variances à fournir. Cela demeure possible grâce au gain d'efficacité de calcul permis par l'approche {rétrograde} de l'AMIS, et permettrait d'envisager différents niveaux de confiance accordés aux observation (pour $\varObs$), ainsi que plusieurs hypothèses sur le profil du rejet (pour $\varQ$).




\section{Exemple d'application à un cas urbain}

Après avoir utilisé le cas-test Beaune pour tester plusieurs aspects de l'algorithme d'estimation, nous travaillons ici dans un cadre différent, qui est celui du milieu urbain. Plusieurs nouveaux facteurs d'incertitude sont ainsi introduits, et le but de ce paragraphe est d'examiner les résultats de l'AMIS dans un contexte à forte complexité.\\

\subsection{Présentation du cas-test}

Nous considérons désormais un cas-test en milieu urbain caractérisé par la présence d'obstacles multiples et de géométries variées sur le domaine (bâtiments). Pour cela, on utilise une reconstitution du quartier parisien de l'Opéra (figure \ref{fig_opera_config}).\\

\begin{figure}[h!]
	\centering
	\begin{subfigure}[t]{0.5\textwidth}
		\centering
		\includegraphics[width=1\textwidth]{opera_config_domaine.png}
		\caption{}
		\label{fig_opera_config}
	\end{subfigure}%
	\begin{subfigure}[t]{0.5\textwidth}
		\centering
		\includegraphics[width=0.75\textwidth]{opera_osm.png}
		\caption{}
		\label{fig_opera_carte}
	\end{subfigure}
	\caption{A gauche: illustration du milieu bâti, du réseau de capteurs (en noir) et de la source (en magenta) utilisés pour le cas-test Opéra. A droite: carte OpenStreetMap du cas-test Opéra.}
	\label{fig_opera_presentation}
\end{figure}

\subsubsection{Caractéristiques du domaine}
Le domaine couvre une surface de \SI{808}{\meter} $\times$ \SI{882}{\meter}, avec une source unique et un réseau de 10 capteurs, disposés de façon non-régulière. Pour la simulation, le domaine est discrétisé en une grille de $404 \times 441$ mailles, avec une résolution du maillage en $x$ et $y$ de \SI{2}{\meter}: il s'agit d'un espace d'étude plus petit que celui du cas-test Beaune, mais de résolution plus élevée, afin d'assurer une meilleure précision des calculs de dispersion en présence d'une topographie complexe. \\


\subsubsection{Paramètres météorologiques}
Pour ce cas-test, on choisit des paramètres météorologiques instationnaires, à savoir un vent de vitesse constante (\SI{3}{\meter \per \second}), mais dont la direction change toutes les heures:

\begin{center}
	\begin{tabular}{cccc}
		\centering
		Heure & 11:00 &  12:00 &  13:00\\ 
		\hline
		Direction du vent & $230\degres$ & $180\degres$ & $45\degres$
	\end{tabular} 
\end{center}

La combinaison de ces variations temporelles ainsi que de la présence d'obstacles sur le domaine fait que les champs de vent 3D diagnostiqués par SWIFT sont relativement complexes, comme l'illustre la figure \ref{fig_opera_vent}.

\begin{figure}[h!]
	\centering
	\begin{subfigure}[t]{0.5\textwidth}
		\centering
		\includegraphics[width=1\textwidth]{opera_vent_11_00.png}
		\caption{}
		\label{opera_vent_11_00}
	\end{subfigure}%         	
	\begin{subfigure}[t]{0.5\textwidth}
		\centering
		\includegraphics[width=1\textwidth]{opera_vent_12_00.png}
		\caption{}
		\label{opera_vent_12_00}
	\end{subfigure}
	\begin{subfigure}[t]{0.5\textwidth}
		\centering
		\includegraphics[width=1\textwidth]{opera_vent_13_00.png}
		\caption{}
		\label{opera_vent_13_00}
	\end{subfigure}%
	\caption{Champs de vent à \SI{2}{\meter} du sol produits par SWIFT aux trois échéances météorologiques considérées}
	\label{fig_opera_vent}
	
\end{figure}



\subsubsection{Capteurs, source et simulation des observations}
Le domaine contient un réseau de 10 capteurs, qui sont placés aux centres de diverses intersections de rues, ainsi que sur des places publiques. Ils sont situés à une hauteur de 2m, et délivrent des valeurs de concentrations moyennées sur des plages de 5 minutes entre 11h35 et 13h.

La source est également située à 2m du sol, positionnée au niveau d'une grande intersection, et émet un rejet bref d'une durée de 10 minutes entre 12h10 et 12h20, avec un débit constant de $10^4$ unités/s. Ces caractéristiques se rapprochent de celles d'un rejet d'origine malveillante, par exemple suite à l'explosion d'une "bombe sale". 

On suit le même raisonnement que pour le cas-test Beaune et on simule le vecteur d'observations à partir d'une matrice source-récepteur \textit{backward} afin d'obtenir les mesures simulées de la figure \ref{fig_opera_obs}. Les dimensions de cette source sont celles d'une maille du domaine, à savoir un volume de 2m $\times$ 2m $\times$ 2m.

\begin{figure}[h!]
	\centering
	\includegraphics[width=0.8\textwidth]{concentrations_opera.png}
	\caption{Cas-test Opéra: concentrations mesurées aux capteurs}
	\label{fig_opera_obs}
\end{figure}

Etant donné la taille réduite du domaine ainsi que les conditions météorologiques instationnaires auquel le cas-test est confronté, on observe que malgré un nombre total de capteurs inférieur à celui du cas-test Beaune, la proportion de récepteurs mesurant une concentration non-nulle est ici plus élevée (figure \ref{fig_opera_obs}).


\subsection{Initialisation optimisée de la loi de proposition}

Avec le cas-test Opéra, il est vite apparu que sans apport préalable d'information, l'algorithme aurait des difficultés à converger vers une solution correcte étant donné la complexité {de la configuration à traiter (milieu obstrué et météorologie variable)}. Une procédure préliminaire a donc été mise en place {visant à prédéterminer une zone à favoriser pour la localisation de la source. Pour cela, le but est } d'initialiser les paramètres de la loi de proposition de l'AMIS autrement que par une simple répartition uniforme des particules sur le domaine. {Un ajustement de ces paramètres} en amont de l'exécution de l'AMIS permettrait alors d'explorer des zones potentiellement intéressantes plus rapidement, augmentant ainsi l'efficacité de l'algorithme d'estimation. \\

\newpage

 Pour cela, nous proposons d'utiliser le résultat des calculs de rétro-propagation en modélisant par une mixture de $K$ gaussiennes la densité de probabilité sur la localisation obtenue par rétro-propagation à $D/K$ instants choisis uniformément répartis sur un intervalle $[t_0, t_{T_s}]$, où $t_0$ est l'instant durant lequel est observé le maximum de concentration et {$t_{T_s}$ }est l'instant final de la rétro-propagation, défini par l'utilisateur. Plus précisément, pour un instant {$t_l \in [t_0,t_{T_s}]$}, en notant $\VecTheta = (x,y)$ un point du domaine maillé suivant une grille de dimensions $(N_x,N_y)$, on considère la distribution suivante:

\begin{equation}
p(\VecTheta | t_l,\VecObs) = \sum\limits_{i=1}^{N_x} \sum\limits_{j=1}^{N_y} \widetilde{\omega}_{x_i,y_j}^l ~  \mathcal{U}_{[x_{i-1},x_i]\times [y_{j-1},y_j]}(\VecTheta)
\label{eq_carte2d}
\end{equation}

où :
\begin{itemize}
	\item pour tout point $(x_i,y_j)$ avec $1\leq x_i \leq N_x$ et $1 \leq y_j \leq N_y$, la pondération $\widetilde{\omega}_{x_i,y_j}^l$ est définie par:
	\begin{equation}
		\begin{split}
			\omega_{x_i,y_j}^l &= \max\left(0, \sum\limits_{k=1}^{N_c} \mathbbm{1}(C^*([x_i,y_j] | R_k,t_l) \geq \varepsilon_{RP}) \times \text{sign}(\mathbb{E}({\VecObs_k}) - \varepsilon_{RP})\right) \\
			\widetilde{\omega}_{x_i,y_j}^l &= \dfrac{\omega_{x_i,y_j}^l}{\sum\limits_{i=1}^{N_x}\sum\limits_{j=1}^{N_y}\omega_{x_i,y_j}^l}
		\end{split}
		\label{eq_ponderation}
	\end{equation}
	$\widetilde{\omega}_{x_i,y_j}^l$ {peut être vu comme} la probabilité que la source soit dans la zone définie par les segments $[x_{i-1},x_i]$ et $[y_{j-1},y_j]$ si l'instant d'émission se situait dans l'intervalle $[t_{l-1},t_l]$. Cette grandeur est obtenue via un calcul de rétro-propagation des concentrations conjuguées $C^*([x_i,y_j] | R_k,t_l) $ pour chaque capteur {$R_k$} mesurant les plus grandes valeurs de concentrations. L'équation \eqref{eq_ponderation} revient à comptabiliser le nombre de rétro-propagations en un point de l'espace et du temps si la valeur moyenne mesurée par les capteurs est supérieure à un seuil $\varepsilon_{RP}$ prédéfini. {$\VecObs_k$ désigne l'ensemble des observations fournies par le capteur $R_k$.}
	\item $\mathcal{U}_{[x_{i-1},x_i]\times [y_{j-1},y_j]}(\VecTheta)$ est la loi de probabilité uniforme en deux dimensions sur la surface définie par les segments $[x_{i-1},x_i]$ et $[y_{j-1},y_j]$ . \\
\end{itemize}

L'objectif de la procédure d'initialisation consiste à adapter une mixture de $K$ gaussiennes sur la distribution \eqref{eq_carte2d}, cette mixture pouvant alors s'écrire : 
\begin{equation}
\begin{split}
\psi_{\alpha, \nu}(\VecTheta) &= \sum\limits_{k=1}^K \alpha_k \mathcal{N}(\VecTheta | \nu_k) \\
\nu_k &= (\VecMu_k, \MatSigma_k)
\end{split}
\end{equation}

Pour cela, on choisit de minimiser la divergence de Kullback-Leibler, la procédure d'optimisation de ce critère permettant de déterminer les paramètres recherchés pour la loi de proposition. En appliquant la définition de l'équation \eqref{eq_definition_KL}, on peut écrire la forme explicite de cette divergence: 

\begin{equation}
D\left(p(\VecTheta|t_l,\VecObs) ~ || ~ \psi_{\alpha, \nu}(\VecTheta)\right) = \int \log \left(\dfrac{p(\VecTheta|t_l,\VecObs)}{ \psi_{\alpha, \nu}(\VecTheta)}\right) \psi_{\alpha, \nu}(\VecTheta) d\VecTheta
\label{eq_KL_fitting}
\end{equation}

Minimiser \eqref{eq_KL_fitting} revient de façon équivalente à maximiser le terme suivant : 
\begin{equation}
\argmax_{\alpha_k, \nu_k} \int \log \left(\sum\limits_{k=1}^K \alpha_k \mathcal{N}(\VecTheta | \nu_k) \right)p(\VecTheta|t_l,\VecObs) d\VecTheta
\end{equation}

Cette maximisation ne peut pas se faire de façon analytique, mais il reste possible d'appliquer un raisonnement similaire à celui suivi pour établir les équations \eqref{eq_KL_pour_EM} à  \eqref{eq_maj_gaussien} en résolvant le problème de façon itérative à la façon d'un algorithme EM. Ainsi, à l'itération $m$ de l'algorithme de maximisation, en définissant : 

\begin{equation}
\rho_k (\VecTheta | \alpha_k^m,\nu_k^m) = \dfrac{\alpha_k^m \mathcal{N}(\VecTheta|\nu_k^m)}{\sum\limits_{k=1}^K \alpha_k^m \mathcal{N}(\VecTheta|\nu_k^m)}
\end{equation}

on peut écrire les règles de mise à jour des paramètres de la $k$-ième mixture de la façon suivante : 

\begin{equation}
\begin{split}
\alpha_k^{m+1} &= \sum\limits_{i=1}^{N_x} \sum\limits_{i=1}^{N_y} {\widetilde{\omega}_{x_i, y_j}^l} \rho_k (\VecTheta | \alpha_k^m,\nu_k^m) \\
\VecMu_k^{m+1} &= \dfrac{\sum\limits_{i=1}^{N_x} \sum\limits_{i=1}^{N_y} {\widetilde{\omega}_{x_i,y_j}^l} \rho_k (\VecTheta | \alpha_k^m,\nu_k^m) \VecTheta^T}{\alpha_k^{m+1}} \\
\MatSigma_k^{m+1} &= \dfrac{\sum\limits_{i=1}^{N_x} \sum\limits_{i=1}^{N_y} {\widetilde{\omega}_{x_i,y_j}^l} \rho_k (\VecTheta | \alpha_k^m,\nu_k^m) (\VecTheta - \VecMu_k^{m+1} )(\VecTheta - \VecMu_k^{m+1} )^T}{\alpha_k^{m+1}}
\end{split}
\end{equation}

La figure (\ref{fig_opera_fitting}) présente le résultat de l'initialisation avec l'ensemble des données de rétro-propagation utilisées (figure \ref{opera_carte_fitting}) pour déterminer les paramètres de la loi de proposition, ainsi que la densité de probabilité résultante  (figure \ref{opera_resultat_fitting}), qui sert donc de point de départ pour l'algorithme AMIS. On observe déjà que la zone à privilégier est bien au voisinage de la source à retrouver.\\

\begin{figure}[h!]
	\centering
	\begin{subfigure}[t]{0.5\textwidth}
		\centering
		\includegraphics[width=1\textwidth]{opera_carte_fitting.png}
		\caption{}
		\label{opera_carte_fitting}
	\end{subfigure}%
	\begin{subfigure}[t]{0.5\textwidth}
		\centering
		\includegraphics[width=1\textwidth]{opera_resultat_fitting.png}
		\caption{}
		\label{opera_resultat_fitting}
	\end{subfigure}
	\caption{Illustration de la procédure d'initialisation optimisée de la loi de proposition, avec la carte des rétro-concentrations (gauche) et la densité de probabilité suivant les paramètres initialisés (à droite)}
	\label{fig_opera_fitting}
	
\end{figure}

\subsection{Résultats}

La figure \ref{fig_opera_amis} présente le résultat d'un \textit{run} de 10 itérations de l'AMIS avec 100 particules tirées par itération, avec une initialisation optimisée et en utilisant les paramètres $\varObs = 7 \times 10^{-2}$ et $\varQ = 5\times 10^7$. Comme {prévu}, après l'ajustement initial de la loi de proposition, la localisation de la source est {satisfaisante}, de même, le profil de rejet est bien estimé. Les scores d'erreur sont de $r_d = 0.048$ pour l'estimation ponctuelle et $Err(\widetilde{\VecMu}_q) = 628.974$ pour le débit.


\begin{figure}[h!]
	\centering
	\begin{subfigure}[t]{0.5\textwidth}
		\centering
		\includegraphics[width=1\textwidth]{opera_amis_kde_x.png}
		\caption{Position en $x$}
		\label{opera_amis_x}
	\end{subfigure}%
	\begin{subfigure}[t]{0.5\textwidth}
		\centering
		\includegraphics[width=1\textwidth]{opera_amis_kde_y.png}
		\caption{Position en $y$}
		\label{opera_amis_y}
	\end{subfigure}
	\begin{subfigure}[t]{0.65\textwidth}
		\centering
		\includegraphics[width=1\textwidth]{opera_amis_Q.png}
		\caption{Profil d'émission avec intervalle de confiance à  $\pm 2 \widetilde{\sigma}_q^2$ (gris)}
		\label{opera_amis_q}
	\end{subfigure} 
	\caption{Résultats de l'algorithme d'estimation sur le cas-test Opéra avec une initialisation optimisée}
	\label{fig_opera_amis}
\end{figure}

Bien que l'initialisation optimisée ait permis une { reconstruction correcte} du terme source, il est à noter que la procédure d'échantillonnage depuis la loi de proposition elle-même n'est pas complètement efficace dans le cas urbain. En effet, durant cette procédure, un certain nombre de particules est tiré à l'intérieur des bâtiments, or une des hypothèses de départ stipule que la source se situe en extérieur. Pour toute particule tirée dans un bâtiment, la rétro-concentration qui lui est attribuée est automatiquement nulle, ce qui par conséquent réduit considérablement le nombre de particules pouvant avoir un poids d'importance représentatif, et réduit ainsi l'efficacité du processus d'échantillonnage d'importance. 

Une solution envisagée a été d'implémenter le ré-échantillonnage systématique de toute particule tirée dans un bâtiment, tant que celle-ci n'est pas en extérieur. Cependant, une telle démarche peut grandement allonger le temps de calcul, car la surface couverte par les bâtiments est supérieure à celle {de la partie extérieure du domaine de simulation}. De plus, l'échantillon obtenu après ces ré-échantillonnages ne représente plus mathématiquement la loi de proposition, et fausse donc le déroulement de l'algorithme d'estimation. 

Une autre alternative plus simple consisterait à augmenter le nombre de particules par itération, afin d'accroître la proportion d'échantillons tirés en extérieur. {Grâce au pré-calcul des rétro-concentrations, la charge supplémentaire de calcul générée par une quantité plus importante de particules reste raisonnable, bien moins élevée que si l'on avait voulu faire la même chose avec une approche utilisant un modèle direct.}\\

{
Ce chapitre a permis d'illustrer les apports de l'utilisation en mode rétrograde du modèle de dispersion. La génération des matrices source-récepteur à partir de données de rétro-concentration pré-calculées a permis une utilisation optimale du système PMSS, ne nécessitant pas de multiples calculs de dispersion. Les exemples d'application sur lesquels cette approche a été utilisée ont montré des résultats corrects de l'estimation des paramètres de la source, à la fois sur sa localisation et sur la reconstruction de son débit d'émission. Le cas-test Beaune a permis d'illustrer l'influence des paramètres $\varObs$ et $\varQ$, ainsi que de souligner l'importance prise par un capteur suffisamment proche de la source pour obtenir une bonne estimation. Le cas-test Opéra a confronté l'algorithme à une situation plus complexe, mais qui a pu être bien traitée grâce à l'initialisation améliorée de l'AMIS, en réutilisant les résultats de rétro-propagation afin d'isoler une zone d'intérêt sur laquelle sont ajustés les paramètres de départ de la loi de proposition.\\}

Une question importante est toutefois soulevée par ces résultats, et concerne le réglage de $\varObs$ et $\varQ$. Dans ce chapitre, les valeurs par défaut choisies pour les différents cas-tests ont été obtenues par essais et erreurs successifs, mais en pratique il est nécessaire de développer une procédure mieux élaborée. Une solution serait de considérer ces hyperparamètres comme des \textit{variables latentes}, afin de leur affecter une loi de probabilité a priori, et ainsi de considérer l'ensemble du processus d'estimation comme un \textit{modèle hiérarchique bayésien} (voir par exemple les récents travaux présentés dans \cite{Smidl2015}).





