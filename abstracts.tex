\thispagestyle{plain}

\begin{flushright}
\textbf{{ Adaptive Bayesian inference for source term estimation in atmospheric dispersion}}
\end{flushright}
{\small \textbf{Abstract}: In atmospheric physics, reconstructing a pollution source is a challenging but important question : it provides better input parameters to dispersion models, and gives useful information to first-responder teams in case of an accidental toxic release.
Various methods already exist, but using them requires an important amount of computational resources, especially as the accuracy of the dispersion model increases. A minimal degree of precision for these models remains necessary, particularly in urban scenarios where the presence of obstacles and the unstationary meteorology have to be taken into account. One has also to account for all factors of uncertainty, from the observations and for the estimation. 
The topic of this thesis is the construction of a source term estimation method based on adaptive Bayesian inference and Monte Carlo methods. First, we describe the context of the problem and the existing methods. Next, we go into more details on the Bayesian formulation, focusing on adaptive importance sampling methods, especially on the AMIS algorithm. The third chapter presents an application of the AMIS to an experimental case study, and illustrates the mechanisms behind the estimation process that provides the source parameters’ posterior density. Finally, the fourth chapter underlines an improvement of how the dispersion computations can be processed, thus allowing a considerable gain in computation time, and giving room for using a more complex dispersion model on both rural and urban use cases. \\

\textbf{Mots-clés}: Inférence bayésienne, Méthodes de Monte-Carlo, Techniques adaptatives, Physique de l'atmosphère, Estimation du terme source. \\

}
\rule{\linewidth}{.5pt}
\begin{flushright}
	\textbf{{Inférence bayésienne adaptative pour la reconstruction de source en dispersion atmosphérique}}
\end{flushright}
{\small \textbf{Résumé}:En physique de l’atmosphère, la reconstruction d’une source polluante à partir des mesures de capteurs est une question importante. Elle permet en effet d’affiner les paramètres des modèles de dispersion servant à prévoir la propagation d’un panache de polluant, et donne aussi des informations aux primo-intervenants chargés d’assurer la sécurité des populations.
Plusieurs méthodes existent pour estimer les paramètres de la source, mais leur application est coûteuse à cause de la complexité des modèles de dispersion. Toutefois, cette complexité est souvent nécessaire, surtout lorsqu’il s’agit de traiter des cas urbains où la présence d’obstacles et la météorologie instationnaire imposent un niveau de précision important. Il est aussi vital de tenir compte des différents facteurs d’incertitude, sur les observations et les estimations.
Les travaux menés dans le cadre de cette thèse ont pour objectif de développer une méthodologie basée sur l’inférence bayésienne adaptative couplée aux méthodes de Monte Carlo pour résoudre le problème d’estimation du terme source. Pour cela, nous exposons d’abord le contexte scientifique du problème et établissons un état de l’art. Nous détaillons ensuite les formulations utilisées dans le cadre bayésien, plus particulièrement pour les algorithmes d’échantillonnage d’importance adaptatifs. Le troisième chapitre présente une application de l’algorithme AMIS dans un cadre expérimental, afin d’exposer la chaîne de calcul utilisée pour l’estimation de la source. Enfin, le quatrième chapitre se concentre sur une amélioration du traitement des calculs de dispersion, entraînant un gain important de temps de calcul à la fois en milieu rural et urbain.\\

\textbf{Keywords}: Bayesian inference, Monte-Carlo methods, Adaptive methods, Atmospheric physics, Source term estimation.
}


\pagebreak