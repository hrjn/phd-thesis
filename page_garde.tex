% - - - - - - - début de la page 
\thispagestyle{empty}


{\large

\begin{center}

%{\bf TH\`ESE DE DOCTORAT DE \\ l'UNIVERSIT\'E LILLE 1}
{\bf Université Lille 1 - Sciences et Technologies\\ 
	Ecole Doctorale des Sciences Pour l'Ingénieur} 

\vspace*{0.4cm}

{\large {\bf TH\`ESE}}

\vspace*{0.25cm}

présentée en vue d'obtenir le grade de:

\vspace*{0.25cm}

{\large {\bf DOCTEUR}}

\vspace*{0.25cm}

Spécialité: Automatique, Génie Informatique, Traitement du Signal et de l'Image

\vspace*{0.4cm}

Par:

\vspace*{0.25cm}

{\Large {\bf Harizo RAJAONA}}

\vspace*{1.5cm}



{\huge {\bf Inférence bayésienne adaptative pour la reconstruction de source en dispersion atmosphérique }}

\end{center}
\vspace*{1.5cm} 

}
\begin{flushleft}

Soutenue le 21 novembre 2016 devant le jury composé de:\\[2ex]



\addtolength{\textwidth}{-1cm}
\begin{tabular}{lll}
 M. François {\sc Septier} & Maître de Conférences, Télécom Lille / CRIStAL & Directeur de thèse\\
 M. Thierry {\sc Chonavel} & Professeur, Télécom Bretagne & Rapporteur \\
 M. Hichem {\sc Snoussi} & Professeur, Université de Technologie de Troyes &  Rapporteur  \\
 M. Lionel {\sc Soulhac} & Maître de Conférences, Ecole Centrale de Lyon & Examinateur  \\
 M. Patrick {\sc Armand} & Expert senior, CEA &  Examinateur  \\
 M. Patrick {\sc Bas} & Directeur de Recherche CNRS, Ecole Centrale de Lille &  Examinateur  \\
 M. Armand {\sc Albergel} & Directeur Général Délégue, ARIA Technologies &  Invité  \\
\end{tabular}
\addtolength{\textwidth}{1cm}

\end{flushleft}


% - - - - - - - fin de la page 
\pagebreak