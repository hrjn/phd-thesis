\section*{Remerciements}

Mes premiers remerciements vont au professeur Yves Delignon, qui fût le directeur originel de cette thèse et qui n'est malheureusement plus là pour en voir l'aboutissement. Ses conseils, sa patience, et son enseignement de la rigueur et de la persévérance ont été indispensables pour achever ce travail, et pour cela je lui en suis plus que reconnaissant. \\

Un grand merci à François Septier, mon co-encadrant académique, puis directeur sur les derniers mois de la thèse, pour m'avoir accompagné et guidé durant ces années. Malgré la distance, tu as toujours été disponible pour répondre à mes questions, et surtout tu as su trouver les mots pour m'aider le cap dans les moments difficiles. (...)

Je remercie M. Patrick Armand de m'avoir donné l'opportunité de travailler sur ce sujet de thèse en acceptant de m'accueillir au sein de son laboratoire au CEA. Je remercie également tous les collègues du labo pour tous les bons moments passés ensemble,  (...) Mention spéciale pour mes "voisins de palier" chimistes, pour les discussions autour du café le matin (sans oublier la lecture de l'horoscope!), pour leur bonne humeur et leur soutien. J'ai aussi beaucoup appris grâce à vous (...)



Mes dernières pensées vont vers ma famille: mes parents et ma soeur, qui ont toujours été là pour moi, et qui n'ont jamais cessé de croire en moi, même dans mes moments de doute les plus profonds. Cette oeuvre de tant de jours en un jour accomplie est aussi la vôtre. Et parce que sans vous je ne serai pas allé bien loin, parce qu'on a toujours voulu vivre les choses importantes ensemble, je vous dédie ce manuscrit. Je vous aime, et j'espère pouvoir vivre beaucoup d'autres aventures avec vous. Ensemble. 

\newpage

\setlength{\epigraphwidth}{0.8\textwidth}
\epigraph{Mais je connais la solitude. Trois années de désert m'en ont enseigné le goût. \\
	
	On ne s'y effraie point d'une jeunesse qui s'use dans un paysage minéral, mais il y apparait que, loin de soi, c'est le monde entier qui vieillit. Les arbres ont formé leurs fruits, les terres ont sorti leur blé, les femmes déjà sont belles. Mais la saison avance et l'on est retenu au loin... \\
	
	Et les biens de la terre glissent entre les doigts comme le sable fin des dunes.}{\textit{Terre des Hommes\\ Antoine de Saint-Exupéry}}
