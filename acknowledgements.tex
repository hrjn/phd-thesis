\clearpage
\pagestyle{empty}


\section*{Remerciements}
{\small 
Mes premiers remerciements vont au professeur Yves Delignon, qui fût le directeur originel de cette thèse et qui n'est malheureusement plus parmi nous pour en voir l'aboutissement. Ses conseils, sa patience, et son enseignement de la rigueur et de la persévérance ont été indispensables pour achever ce travail, et pour cela je lui en suis plus que reconnaissant. \\

Un immense merci à François Septier, qui m'a guidé tout au long de ces années, et qui a assuré la direction de ma thèse sur les derniers mois. Sa patience, son attention et sa disponibilité malgré la distance ont été indispensables au bon déroulement de cette thèse. Il a su trouver les mots pour répondre clairement à chacune de mes questions, mais aussi et surtout, pour m'encourager à franchir les étapes difficiles qui jalonnent la vie d'un doctorant. \\

J'adresse également mes sincères remerciements aux membres du jury pour l'attention qu'ils ont porté à l'évaluation de mon travail. Merci aux professeurs Thierry Chonavel et Hichem Snoussi pour avoir accepté d'endosser le rôle de rapporteur, pour leur relecture minutieuse de mon manuscrit ainsi que pour leurs intéressantes remarques sur son contenu. Merci aussi à MM. Patrick Bas et Lionel Soulhac d'avoir accepté de se déplacer pour assister à la soutenance. \\

Je remercie Patrick Armand pour m'avoir accueilli au sein de son laboratoire au CEA et co-encadré cette thèse: sa vision globale du sujet mais aussi son souci du détail ont toujours permis de faire avancer les choses dans le bon sens. Merci également à Christophe M.  et Jean A., qui ont toujours été à l'écoute des doctorants du service, et à Karine P. et Emilie D. pour leur assistance dans les (nombreuses!) démarches administratives du CEA.

 Merci évidemment à tous les membres du LIRC pour tous les bons moments passés ensemble, à parler boulot mais aussi foot ou encore robotique. Merci à Marguerite M., experte en titre et grande voyageuse du labo, et à Maud L.-W. Merci à Christophe D., expert en questions informatiques mais aussi imbattable sur l'actualité sportive. Merci à Luc P., en qui j'ai trouvé un hobbyiste tout aussi \textit{geek} que moi. Merci à Sylvia S., Pascal A. et Thomas A., qui avaient toujours de quoi rire ou plaisanter pour garder la bonne humeur. Merci à Christelle C., voisine du sud qui a sû ne pas perdre l'accent. Merci à Jean-Baptiste S. et Thomas L., co-explorateurs (on sait ce qu'il y a au dernier étage!) et jardiniers improvisés du fameux "loft", dont je n'oublie pas tous les occupants que j'ai eu le plaisir de connaître.
 

 
J'ai une pensée spéciale aux collègues chimistes qui ont accepté de me faire une place dans leurs locaux, et avec qui j'ai eu le plaisir de discuter, plaisanter, et apprendre au quotidien: sans votre gentillesse et votre bonne humeur, cette thèse n'aurait pas été la même. Merci à Françoise Z. pour tous les cookies ramenés d'Angleterre. Merci à Françoise L. pour ses récits épiques de voyage au bout du monde. Merci à Frédéric P., qui a encore des efforts à faire en informatique :-) mais qui compense très largement par sa connaissance sans faille des musées de Paris. Merci à Maxime B., qui a autrefois vécu outre-Atlantique, et qui m'a convaincu d'y aller faire un tour un de ces jours pour visiter. 

Impossible également d'oublier tous ceux avec qui j'ai partagé le fameux bureau 274, et qui ont transformé cet espace de travail en un lieu de partage et d'échanges. Merci à Yasmine,  doctorante pionnière du laboratoire, pour toutes les discussions qu'on a partagées, parfois très sérieuses, parfois un peu moins, mais toujours avec quelque chose à apprendre à la clé. Merci à Sébastien S., chimiste de la pollution de l'air et comme moi, grand amateur des oeuvres de Nobuo Uematsu. Merci à Sébastien V. qui a sû mettre de la couleur (de l'orange en particulier) et de la musique dans la vie du bureau. Merci à Christophe, le vétéran du bureau, pour avoir partagé son expérience de chercheur et pour ses conseils dont je saurai me rappeler à l'avenir. Merci à Robin L. pour ses explications sur l'assimilation de données et son lien avec mon sujet de recherche. Merci à Adrien et Daniel, autrefois stagiaires, et maintenant doctorants: je vous souhaite tout le meilleur pour la fin de la thèse et pour la suite. 

J'ai eu la chance de travailler sur un sujet de recherche appliquée, et donc de passer un certain temps au sein de la PME ARIA Technologies pour implémenter une grande partie de mes travaux. Je souhaite donc remercier ses dirigeants et co-fondateurs Jacques Moussafir et Armand Albergel pour la confiance qu'ils m'ont accordé,  ainsi que pour leur présence et leurs avis éclairés lors des comités de thèse, malgré leurs agendas surchargés respectifs. Un grand merci à Christophe Olry, dont l'aide s'est révélée indispensable lorsqu'il a fallu "brancher" les algorithmes d'estimation sur PMSS, merci également à Laurent Makké qui a pris sa suite, et dont l'implication a permis d'obtenir les derniers résultats du cas urbain. Merci également à Jérôme, Maxime, Cyrille et Félix pour leurs conseils. Plus généralement, je remercie tous les "Ariotes" pour leur accueil chaleureux durant les nombreux jours passés à Boulogne-Billancourt, vous avez chacun contribué à votre façon à la partie "pratique" de cette thèse. \\

Terminer une thèse, c'est aussi la fin du chapitre "scolaire" d'une vie, et mon parcours n'aurait pas été ce qu'il est sans ceux qui m'ont appris et inspiré durant toutes ces années. Merci à Caroline Paulus et à George Malliaras, mes premiers mentors dans la Recherche, pour m'avoir mis le pied à l'étrier et initié à cet univers que je connais à présent un peu mieux. Merci à mes professeurs de l'ENSEA, en particulier Sophie Olijnyk et Thomas Tang, ainsi qu'au légendaire professeur Kasbari, dont le cours de physique des composants restera pour moi la meilleure expérience pédagogique. Merci à Christophe Coste pour m'avoir donné la chance que je n'attendais plus. Enfin, merci à Eugénie Gaudin et Alain Tertzaguian, avec qui tout a commencé. \\

Merci du fond du coeur à mes amis dont la présence et le soutien ont beaucoup aidé, avec une pensée particulière pour les camarades de promo ENSEArques, ainsi que pour les incontournables HDP qui ont toujours été là quand il fallait. \\

Pour terminer, je souhaiterais dire merci à ma famille, aux oncles, tantes et cousins qui n'auront bientôt plus à demander "comment vont les études?". Les derniers mots seront pour mes parents et ma soeur: c'est parce que vous n'avez jamais cessé de croire en moi que je suis allé aussi loin. Cette réussite, c'est aussi et surtout la vôtre. Vous êtes ma fierté, et ce manuscrit vous est dédié.
}




